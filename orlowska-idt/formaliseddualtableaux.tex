\documentclass[a4paper]{article}
\usepackage{bussproofs}
\usepackage{amsthm}
\usepackage{amssymb}
\usepackage{url}

\newtheorem{theorem}{Theorem}
\newtheorem{lemma}{Lemma}
\newtheorem{definition}{Definition}
\newtheorem{corollary}{Corollary}

\newcommand{\hol}{\texttt{HOL4}}

\newcommand{\ie}{\textit{i.e. }}

\newcommand{\tsim}{\textasciitilde}
\newcommand{\ttsim}{\texttt{\textasciitilde}}

\renewcommand\labelenumi{(\roman{enumi})}
\renewcommand\theenumi\labelenumi

\newcommand\comment[1]{}

\title{Machine-checked Meta-theory of Dual-Tableaux for Intuitionistic
Logic}
\author{Jeremy E Dawson and Rajeev Gor\'e
\\ Research School of Computer Science
\\  The Australian National University}

\date{}

\begin{document}
\maketitle

\begin{abstract}
  We describe how we formalised the meta-theory of Melvin Fitting's
  dual-tableaux calculi for intuitionistic logic using the \hol{}
  interactive theorem prover. The paper is intended for readers
  familiar with dual-tableaux  who might be interested in, but daunted
  by, the idea of
  formalising the required notions in a modern interactive theorem
  prover. 
\end{abstract}  

\tableofcontents

\newpage
\section{Introduction and motivation}

% \textbf{Ensure that Mel cites our work properly. He mentions ``proof
%   theory'' in his citation on page 3 line 5, but this is not really
%   accurate as our contribution mimics his in using the semantics. His
%   citation should also say ``Dawson and Gor\'e'' not just ``Gor\'e''.}

Tableaux calculi originated with the work of Beth in the
1950s~\cite{beth-padoas}. In 1959, they were used by
Kripke~\cite{kripke-completeness} to prove the soundness and
completeness of his semantics for modal logics such as S5.  They were
extended to many modal logics and to intuitionistic logic by Melvin
Fitting~\cite{fitting-proof} in the 1970s. Since 1992, they have
gained a life of their own via the conference series named ``Theorem
Proving with Analytic Tableaux and Related Methods'' which will have
its 25th anniversary in Brazil in 2017. However, an honest appraisal
of the literature must acknowledge a parallel, and sometimes more
advanced, tradition known as ``dual-tableaux'' which arose in Poland
from the work of Rasiowa, Sikorski and Ewa
Orlowska~\cite{orlowska-joanna-book}.  Here, we pay homage to that
tradition.

Over the past decade, we have formalised many aspects
of proof-theory including sequent
calculi~\cite{DBLP:conf/lpar/DawsonG10} and display
calculi~\cite{dawson-gore-formalised-cut-admissibility}. Here, we turn
our attention to the meta-theory of dual-tableaux
calculi. Specifically, we show how to formalise the semantic soundness
and completeness proofs for the dual-tableaux
calculi for intuitionistic logic given by Melvin
Fitting~\cite{fitting-dual-tableau}.  Our hope is that it will serve
as a guide to others who may want to follow in our footsteps.

We assume that the reader is familiar with Fitting's
chapter~\cite{fitting-dual-tableau} in this volume, but not familiar
with interactive theorem proving. All of our \hol{} files can be found
here: % \marginpar{see text}
\url{http://users.cecs.anu.edu.au/~jeremy/hol/idt/}
and are also available on GitHub at 
\url{https://github.com/jeremydaw/idt} in the directory \texttt{hol}
together with a \texttt{README.md} file detailing the HOL version used
and instructions for compiling and running the proof files.

\section{\hol: an interactive theorem prover}

We chose to work with the interactive theorem prover called
\hol~\cite{DBLP:conf/tphol/Gordon08}, which implements Dana Scott's
Logic of Computable Functions~\cite{scott-computable}. The user must
first encode all of the required definitions of the meta-theory into
\hol{} so we provide most of the details of our definitions. The user
then inputs a goal which \hol{} is asked to prove. Typically, \hol{}
reduces that goal into multiple subgoals and expects the user chooses
the next step of the proof to perform. \hol{} accepts the next step
only if it can be used in a sound way to reduce the chosen subgoal
into further subgoals. Thus \hol{} also keeps track of the proof
process and the current stage of the proof is always visible to the
user. Here, we only show our encoding of the various aspects of the
meta-theory of dual-tableaux, and state the lemmata and theorems that
we proved inside \hol.

\subsection{Why should we trust \hol?}

As with many other interactive theorem provers, \hol{} is a trusted
system because its code-base is small, around 4000 lines of
code, is written in a functional programming language ML, and has
been scrutinised by experts in logic and theorem proving over a period
of 40 years. Moreover, \hol{} can produce a proof-script of the final
proof which can be checked by other scrutineers or even other
interactive theorem provers.

\subsection{The syntax of \hol}

The logic implemented by the \hol{} theorem prover is called
classical higher-order logic. It is ``higher-order'' in that functions and
predicates are first-class citizens which can be quantified over and
which can be passed to each other as arguments. The basic operators of
higher-order logic are
these:

\begin{center}
\begin{tabular}[c]{|c|c|c|c|c|c|c|c|}
\hline
 $\top$ & $\bot$ & $\land$ & $\lor$ & $\to$ & $\lnot$ & $\forall$ & $\exists$
\\ \hline
 \texttt{T} & \texttt{F} & \verb!/\! & \verb!\/! & \texttt{==>} & \verb!~! & \texttt{!} & \texttt{?} 
\\ \hline
\end{tabular}
\end{center}


The syntax of \hol{} also provides for writing lists, pairs and various other
data structures which we shall illustrate as needed. 

The logic is typed and so there is a separate syntax for defining new
types from existing base types provided by \hol{} such as \texttt{nat}
and \texttt{int} for the types of natural numbers and integers
respectively. 

For example, the symbol \texttt{\#} is the infix type constructor
denoting the type of pairs and the symbol \texttt{:} is the infix
operator for ``member of type''. Thus, \mbox{\texttt{a :\ alpha}} encodes that
``\texttt{a} is of type \texttt{alpha}''. 
If also \mbox{\texttt{b :\ beta}} then \hol{}
can deduce that \texttt{(a, b) :\ alpha \# beta} which encodes
``\texttt{(a, b)} is in the type that consists of pairs of objects from
types \texttt{alpha} and \texttt{beta} respectively''.  But generally,
type constructors are written postfix, so \texttt{alpha set} is the type
of sets of items of type \texttt{alpha}.

\section{Capturing the syntax of dual-tableaux}

We now describe how we encoded the syntax of dual-tableaux into \hol.
We build the encoding by first encoding the notion of formulae,
then sets of (signed) formulae and then the notions of dual-tableau
rules, and finally the notion of (closed) dual-tableaux.

\subsection{(Unsigned) Formulae}
\label{sec:unsigned-formulae}

Strings prefixed with an apostrophe (\texttt{'}) are treated by
\hol{} as type variables.  Using such a type variable \texttt{'a}, we
first define a datatype for formulae where the type \texttt{'a} of an
atom is a variable left to be chosen later.

\begin{samepage}
\begin{definition}[formula]
 The formulae of intuitionistic logic are built from an infinite
 supply of atomic formulae (\texttt{Atom 'a} over some base type
%\marginpar{Could we put Atom clause first? SOME PROOFS BREAK, WOULD NEED TO FIX}
 \texttt{'a})
 using the connectives
 $\land$ (\texttt{And}),
 $\lor$ (\texttt{Or}),
 $\to$ (\texttt{Imp}), and
 $\lnot$ (\texttt{Not}) as usual:
\begin{verbatim}
datatype formula = And formula formula
                 | Or formula formula
                 | Imp formula formula
                 | Not formula
                 | Atom 'a ;
\end{verbatim}
\end{definition}
\end{samepage}

Thus we get the type \texttt{'a formula} where \texttt{'a} is a type variable:
in text, we use $\alpha, \beta, \ldots$ for type variables.

The effect of this definition is to declare to \hol{} how it can
recognise a string as a formula over the type \texttt{'a}: for
example, the string \texttt{Imp (Atom 1) (Atom 2)} would be recognised
by \hol{} as a formula built out of atoms of type \texttt{num} of
natural numbers since the items in the scope of the string \texttt{Atom}
are all natural numbers.

The strings \texttt{And}, \texttt{Or}, \texttt{Imp}, \texttt{Not} and
\texttt{Atom} are called the constructors of the datatype
\texttt{formula}. Moreover, these constructors are the only way to
construct formulae, hence we can perform induction on the structure of
a formula by starting at the atoms and dealing with a case for each
connective. 

\subsection{Signs and signed formulae}

In \hol{}, there are two pre-defined constants \texttt{T} and
\texttt{F} which make up the pre-defined type \texttt{bool} whose
members are exactly these two constants. By forming a pair
\texttt{(b,f)} where \texttt{b} is of type \texttt{bool} and
\texttt{f} is of type \texttt{'a formula}, we shall use these
constants as the signs of our signed-formulae. Thus the type
\texttt{bool \# 'a formula} contains the set of all pairs where the
first component is one of the ``signs'' \texttt{T} and \texttt{F} and
the second component is a formula.  We then define \texttt{sf} as
a type abbreviation for such pairs, thereby hiding the specifics
of signed formulae.
\begin{definition}[signed formula]\label{def-signed-formula}
  A \emph{signed formula} of type
  \texttt{sf}
  is a pair \texttt{(b, f)} where
  \texttt{b} is either the constant (sign)
  \texttt{T} or else is the constant (sign)
  \texttt{F}, and 
  \texttt{f}
  is a formula:
\begin{verbatim}
   val _ = Parse.type_abbrev ("sf", ``: (bool # 'a formula)``) ;
\end{verbatim}
\end{definition}

This just means that all occurrences of the type \texttt{'a sf} are
expanded to the type \texttt{bool \# 'a formula} making
\texttt{sf} a type operator with one argument \texttt{'a}.
% in the \texttt{bool \# 'a formula} supplied.  Thus \texttt{'b sf} means
% \texttt{bool \# 'b formula}, etc, allowing us to hide the specifics of
% signed formulae.

In the sequel, we sometimes use \texttt{sf} as the name of a
signed-formula and also sometimes use it as the type of signed formula
defined above. Sometimes, we also use \texttt{'sf} to indicate an
arbitrary type variable, which will be instantiated with the type
\texttt{'a sf} when used in our \hol{} proofs. We try to explain these
uses when they occur.

\subsection{Signed-formula sets as unsigned formula set pairs}

For a set $S$ of signed formulae, 
Fitting~\cite[Definition~5]{fitting-dual-tableau} 
defines the two sets 
$S^T = \{T~X \mid T~X \in S \}$ 
and
$S^F = \{F~X \mid F~X \in S \}$.
He also defines the ability to strip the signs and extract only the
unsigned formulae via:
$S^\circ = \{X \mid T~X \in S \mbox{ or } F~X \in S \}$.
%
A set of signed-formulae can also be seen as a sequent built out of
unsigned formulae collecting the \textit{F}-signed formulae on the
left and the \textit{T}-signed formulae on the right (without their
respective signs). For example, the set $\{(F,a), (T,b), (F,c)\}$ of
signed-formulae can be seen as the sequent $a,c \vdash b$ which can be
represented by a pair $(\{a,c\}, \{b\})$ of sets of (unsigned)
formulae. The ability to move to and fro between these two
representations is useful later, so we define functions
\texttt{mk\_seq} and \texttt{dest\_seq} which switch between the two
representations.  We also illustrate some \hol{} syntax.

The construct \texttt{'f} is a type variable: it will stand for a
(unsigned) formula.
In \hol, 
% \marginpar{wording changed}
the type $\alpha\ \mathit{set}$ 
is ``syntactic sugar'' for $\alpha \to \mathit{bool}$.
Whether a particular term $x$ (say) of type 
$\mathtt{\alpha}$ is or
is not in the set $P$ of type $\alpha\ \mathtt{set}$ is determined by the
value \texttt{T} or \texttt{F} from
\texttt{bool} which the predicate $P(x)$ takes.
Separately, $x \in P$ (\texttt{x IN P}) is logically defined to be $P(x)$
(\texttt{P x)}. Although \texttt{x IN P} and $P(x)$ are provably equivalent, and thus
synonymous, they are not identical terms.

The function $\mathit{FST} : \gamma \times \delta \to \gamma$
returns the first component of type $\gamma$ of a pair
of type $(\gamma,\delta)$.
If we put $\gamma = \mathit{bool}$ then 
$\mathit{FST}$ is of type 
$\mathit{bool} \times \delta \to \mathit{bool}$,
which is syntactic sugar for $(\mathit{bool} \times \delta)~\mathit{set}$
in \hol{}.
%thus allowing us to view  $\mathit{FST}$ as a set.
% CHANGED Fri Feb  3 12:48:21 AEDT 2017
For a term $z: \mathit{bool} \times \delta$,
the predicate $FST(z)$ evaluates to true exactly when
the first component of $z$ evaluates to $T$.
Thus $FST : (\mathit{bool} \times \delta)\ \mathit{set}$
means the set of all pairs (of the appropriate type) of the form $(T, x)$.

\begin{definition}
The functions
\mbox{\rm\texttt{get\_ts}} 
and
\mbox{\rm\texttt{get\_fs}}
are defined as:
\\[1em]
\mbox{\rm
  \begin{tabular}[c]{ll}
  \\ \texttt{get\_ts } & \texttt{: (bool \# 'f) set -> 'f set}
  \\ \texttt{get\_ts sfs} & 
           \texttt{= (IMAGE SND (sfs INTER FST))}
  \\       & $= \{f: \mathtt{'f} \mid (T, f) \in sfs\}$
  \\
  \\ \texttt{get\_fs} & \texttt{: (bool \# 'f) set -> 'f set}
  \\ \texttt{get\_fs\ sfs} 
     & \texttt{= (IMAGE SND (sfs INTER (\$\ttsim{} o FST))}
  \\ &  $= \{f: \mathtt{'f} \mid (F, f) \in sfs\}$
  \end{tabular}
  }
\end{definition}

Here, the function \texttt{get\_ts} accepts a set of pairs of type
\texttt{(bool, 'f)}, for any type \texttt{'f} and returns a set of
items of type \texttt{'f}. By giving it an argument of type \texttt{sf
  set}, we will cause \texttt{'f} to become \texttt{'a formula}. We
therefore use the name \texttt{sfs} in the code to stand for the name
of a set of signed-formulae.

The construct
\texttt{INTER}
is the \hol{} symbol for set intersection $\cap$.
Since 
\texttt{sfs} will be type \texttt{'a sf set},
the construct 
\texttt{(sfs INTER X)}
immediately forces 
\texttt{(sfs INTER X)}
to take type \texttt{'a sf set} for any function
\texttt{X}.
Putting $X$ to be $FST$ forces $FST$ to be of type 
$\alpha\ sf \to bool$, effectively putting 
$\gamma$ in the general type of $FST$ to $bool$ as described above.
So \texttt{FST} in the context
\texttt{(sfs INTER FST)}
is the set of all signed-formulae where the first
component, the sign, is the Boolean value
\texttt{T}.
The constructor \texttt{o} stands for ``composition'' so
\texttt{(\$\tsim o FST)} 
is the ``composition'' of 
Boolean negation \ttsim{}
and $FST$, so
\texttt{(\$\tsim o FST)} effectively ``flips'' the required first component 
\texttt{T} to be 
\texttt{F}. 
Thus 
\texttt{(sfs INTER (\$\ttsim o FST))}
is the set of \texttt{F}-signed formulae (pairs) from 
\texttt{sfs}.
The construct \texttt{SND} returns the second component of a pair
while \texttt{IMAGE f Y} returns the result of applying $f$ to each $y
\in Y$.
Thus
\texttt{(IMAGE SND (sfs INTER (\$\tsim o FST))}
is the set of second components of 
the set of \texttt{F}-signed formulae (pairs)
from 
\texttt{sfs}: that is,  the formulae 
\texttt{f} that are signed 
\texttt{F} in 
\texttt{sfs}.

\begin{definition}
 The sets 
 $\mathtt{mk\_seq~ sfs}$ 
 and 
 $\mathtt{dest\_seq~(fs,~ts)}$ 
 are defined as:
 \\[1em]
 \mbox{\rm
 \begin{tabular}[c]{ll}
   \texttt{mk\_seq} & \texttt{: (bool \# 'f) set -> 'f set \# 'f set ;}
 \\
   \texttt{dest\_seq} & \texttt{: 'f set \# 'f set -> (bool \# 'f) set ;}
 \\[1em]
   \texttt{mk\_seq sfs} & \texttt{= (get\_fs sfs, get\_ts sfs)}
 \\[1em]
   & $= \{ (Fs, Ts) \mid
           Fs = \{ f: \mathtt{'f} \mid (F, f) \in sfs\}$ 
        \mbox{ and } 
 \\
      & ~~~~~~~~~~~~~~~~~~$Ts = \{ f: \mathtt{'f} \mid (T, f) \in sfs\}~\}$
 \\[1em]
   \texttt{dest\_seq(fs, ts)}
   & \texttt{= IMAGE(\$, F) fs UNION IMAGE (\$, T) ts}
 \\[1em]
   & $= \{\; (F, f) \;\mid\; f \in fs \;\}
     \cup
     \{\; (T, f) \;\mid\; f \in ts \;\}$
 \end{tabular}
 }
\end{definition}

% mk_seq sfs = (IMAGE SND (sfs INTER ($~ o FST)), IMAGE SND (sfs INTER FST)) ;
% dest_seq (fs, ts) = IMAGE ($, F) fs UNION IMAGE ($, T) ts ;

Here,
\texttt{IMAGE (\$, T) ts}
applies the pair-constructor
\texttt{(\$, T)} to each member $f$ of $ts$,
turning $f$ into the $T$-signed formula $(T, f)$. Similarly,
\texttt{IMAGE (\$, F) fs} turns every member $f$ of $fs$ 
into the $F$-signed formula $(F, f)$.

The function \texttt{mk\_seq} accepts a set of pairs where, in each
pair $(a,b)$, the $a$ is of type \texttt{bool} (a sign) and $b$ is of
type \texttt{'f} (an unsigned formula in the case that \texttt{'f} is
\texttt{'a formula}). It produces a result which is a pair $(L,R)$ of
sets where $L$ (the antecedent) and $R$ (the succedent) 
are both sets of type \texttt{'f set}. That is,
$(L, R)$ is the sequent $L \vdash R$. For some given set $S$ of signed
formulae, Fitting would write these as 
$L = (S^F)^\circ$
and 
%\marginpar{JED please check! LOOKS OK}
$R = (S^T)^\circ$ 
respectively.

% Next we give the \hol{} definition of these functions using several
% functions pre-defined in \hol.  The infix operators \texttt{UNION} and
% \texttt{INTER} represent $\cup$ and $\cap$ of sets. Infix \texttt{o}
% is function composition. The functions \texttt{FST} and \texttt{SND}
% select the first and second members of a pair.  Finally,
% \texttt{IMAGE} $f$ $S$ applies function $f$ to each member of set $S$.

% \begin{verbatim}
% INTER : 'c set -> 'c set -> bool ;
% FST   : 'a # 'b -> 'a ;
% SND   : 'a # 'b -> 'b ;
% \end{verbatim}

% Note that, in \hol, the type $\alpha$ \texttt{set} is simply an
% abbreviation for $\alpha \to$ \texttt{bool}.

% The function $\mathit{FST} : \alpha \times \beta \to \alpha$
% returns the first component of type $\alpha$ of a pair of type $(\alpha,\beta)$.
% Then, $\mathit{FST}$ (as a set) is of type 
% $\mathit{bool} \times \beta \to \mathit{bool}$, that is
% $(\mathit{bool} \times \beta)\ \mathit{set}$, and means
% the set of all pairs (of the appropriate type) of the form $(T, x)$.
% Likewise $\$\!\sim \;\circ\; \mathit{FST}$ 
% is the set of pairs whose first component is \textit{F}.

% In the definition above,
% \texttt{sfs INTER FST} may be best understood by viewing \texttt{sfs} as a
% set of pairs and \texttt{FST} as a predicate, so
% \texttt{sfs INTER FST} is
% $\{p \in sfs \,|\, \texttt{FST}\ p\ \textrm{holds}\}$,
% that is, the set of pairs in \texttt{sfs} whose first component is $T$.

% Then 
% \texttt{IMAGE SND (sfs INTER FST)} is
% the set of second components of pairs in \texttt{sfs INTER FST},
% that is, $\{f \,|\, (T, f) \in \texttt{sfs}\}$.

% Thus, for example, if $\texttt{sfs} = \{(T, f), (F, g), (T, h)\}$,
% then $\texttt{sfs INTER FST} = \{(T, f), (T, h)\}$, and
% $\texttt{IMAGE SND (sfs INTER FST)} = \{f, h\}$.

The function \texttt{dest\_seq} does the reverse: it accepts a pair
$(L,R)$ of sets, each of type \texttt{'f set}, and
produces a result which is a set of pairs where, in each pair $(a,b)$,
the $a$ is of type \texttt{bool} and the $b$ is of type \texttt{'f}. 
When \texttt{'f} is \texttt{'a formula}, this gives us a 
set of signed formulae of type \texttt{bool \# 'a formula}.

% The corresponding term constructor is the infix comma: to write it as a prefix
% operator, we prefix it with a `\$'.\marginpar{What is \$$\sim$ ?
% It's just the function not, of type bool -> bool, 
% Syntactically, $\sim$ is not just an ordinary function, but the notation
% is somehow analogous to infix operators.  Don't know why, but
% $\sim$ f x is parsed as $\sim$ (f x)}
% In HOL, the type of an infix operator is curried.
% Thus we have
% \begin{verbatim}
% ($, :'a -> 'b -> 'a # 'b)
% \end{verbatim}

% \begin{verbatim}
% IMAGE :('a -> 'b) -> ('a -> bool) -> 'b -> bool
% IMAGE :('a -> 'b) -> ('a set) -> 'b set
% \end{verbatim}

% Suppose we are given a pair $(\mathit{fs}, \mathit{ts})$ consisting of the sets
% \textit{fs} of formulae to be signed with an \textit{F} and
% \textit{ts} of formulae to be signed with a \textit{T}.  The term
% \texttt{(\$, F)} is a function that constructs a pair $(F, x)$
% from an argument \textit{x}. Here, `\texttt{\$,}' is the pair constructor
% and binary function `,', normally written infix, 
% written as a prefix, curried, function,
% of type $\alpha \to \beta \to \alpha \times \beta$.
% The term \texttt{IMAGE (\$, F) fs}
% is the result of applying the function \texttt{(\$, F)} to every
% member of the set \textit{fs}, producing a set of signed formulae
% where the sign is \textit{F}. Similarly for \texttt{IMAGE (\$, T) ts}.

We can convert between the two formalisms via the following lemma.
\begin{lemma}
Let \texttt{sf} be a set of signed formulae and let \texttt{sq} be a
sequent (a pair of sets of unsigned formulae). Then 
\texttt{(dest\_seq sq = sf)} iff \texttt{(mk\_seq sf = sq)}.
\begin{verbatim}
     (dest_seq sq = sf) <=> (mk_seq sf = sq)
\end{verbatim}
\end{lemma}

\subsection{Rule-skeletons, contexts and rules}
A dual-tableau rule consists of a single premise,
which is a set of signed formulae,
and multiple conclusions, each of which is a set of signed formulae.
The premise typically has a single principal formula
and each conclusion has possibly multiple side-formulae. There is
usually also a context $S$ which remains unchanged from premise to
conclusions. For example, consider the rule for $F ~ X \lor Y$ shown
below on the left. It consists of a rule-skeleton as shown below on
the right, which is then decorated uniformly by the context $S$ to
give the actual rule shown on the left. Its principal formula is
$F~ X \lor Y$ and its side-formula sets are $\{F~X , F~ X \lor Y \}$
and $\{F~Y , F~ X \lor Y \}$:
\[
\rootAtTop
\AxiomC{$S ; F ~ X \lor Y; F ~ X$}
\AxiomC{$S ; F ~ X \lor Y; F ~ Y$}
\BinaryInfC{$S ; F ~ X \lor Y$}
\DisplayProof
\hspace{0.5cm}
\rootAtTop
\AxiomC{$F ~ X \lor Y; F ~ X$}
\AxiomC{$F ~ X \lor Y; F ~ Y$}
\BinaryInfC{$F ~ X \lor Y$}
\DisplayProof
\]
This describes Fitting's first six rules
\cite[Figure~1.6]{fitting-dual-tableau}.

\begin{definition}[rule skeleton]
  A rule-skeleton of type \texttt{'a rule\_sk} is a pair \texttt{(psk, cssk)}
  consisting of a single item \texttt{psk} of type \texttt{'a} and
  a set \texttt{cssk} of sets of items of type
  \texttt{'a} using the type abbreviation below:
\begin{verbatim}
  val _ = Parse.type_abbrev ("rule_sk", ``: ('a # 'a set set)``) ;
\end{verbatim}
%  val _ = Parse.type_abbrev ("rule", ``: ('a # 'a set)``) ;
\end{definition}

Ensuring  that
\texttt{'a} is always a signed formula type \texttt{'b sf} then restricts
rule skeletons to be over a signed formula and a set of sets of
signed formulae.

For example, consider the rule skeleton for
$F ~ X \lor Y$
shown above at right.
% We write Fitting's $F ~ X \lor Y$ as the term
% \texttt{(F, formula Or X Y)},
% of our signed formula type \texttt{sf}.
The intuition is that the first component
\texttt{psk}
will encode the single formula $F ~ X \lor Y$ in the premise of the
rule-skeleton while 
the second component
\texttt{cssk}
of the pair will encode the set 
$\{\{F ~ X \lor Y, F~ X\}, \{F ~ X \lor Y, F~ Y\}\}$ 
of sets
$\{F ~ X \lor Y, F~ X\}$
and
$\{F ~ X \lor Y, F~ Y\}$
of signed formulae.

\begin{definition}
  A rule of type \texttt{'a rule} is a pair
  \texttt{(p, cs)}
  consisting of a (premise) set
  \texttt{p}
  of type
  \texttt{'a} and a (conclusions) set
  \texttt{cs}
  of terms of type \texttt{'a}. 
%  val _ = Parse.type_abbrev ("rule_sk", ``: ('a # 'a set set)``) ;
\begin{verbatim}
  val _ = Parse.type_abbrev ("rule", ``: ('a # 'a set)``) ;
\end{verbatim}
\end{definition}

The intuition is that the first component
\texttt{p}
will encode the formula set 
$S \cup \{ F ~ X \lor Y \}$ as the premise of the
rule while 
the second component
\texttt{cs}
of the pair will encode the set 
$\{~S \cup \{F ~ X \lor Y, F~ X\}~ , ~S \cup\{F ~ X \lor Y, F~ Y\}~\}$ 
of sets
$S \cup \{F ~ X \lor Y, F~ X\}$
and
$S \cup \{F ~ X \lor Y, F~ Y\}$
of signed formulae.

Thus \texttt{rule\_sk} (rule-skeleton) models the operation that has 
a premise of type \texttt{'a} (here, a signed formula)
and a set of conclusions
% \marginpar{changed wording of this para}
where each conclusion is a set of items of type \texttt{'a} (here,
signed formula), while \texttt{rule} (a rule, with context, as applied) 
models an operation that has a premise
of type \texttt{'a} (here, a set of signed formulae) and a set
of conclusions, each of which is an item of type \texttt{'a} 
(here, a set of signed formulae).

\begin{definition}
  For all $s$ and $st$,
  \texttt{is\_tab\_rule} $(s, st)$
  returns the set of pairs of the form
  $$(\{s\} \cup U, \;\{U \cup t \mid t \in st\})$$
\begin{verbatim}
is_tab_rule : 'sf rule_sk -> 'sf set rule set
!s st U. is_tab_rule (s, st) ({s} UNION U, IMAGE ($UNION U) st) 
\end{verbatim}
\end{definition}
% rules rewritten to avoid use of INSERT 
% ONCE_REWRITE_RULE [INSERT_SING_UNION] is_tab_rule_rules ;

Intuitively, the relation
\texttt{is\_tab\_rule (s, st)} applies a context \texttt{U} to
the premise \texttt{s} and to each branch of the conclusion
\texttt{st} of such a rule-skeleton.
Here we choose to write the type variable as \texttt{'sf}, to suggest that,
while \texttt{is\_tab\_rule} can be used for any type, we will use it
for the signed formula type.
Also \texttt{A UNION B} encodes $A \cup B$.
The \texttt{!} is the universal quantifier $\forall$,
while \texttt{?} is the existential quantifier $\exists$.
The function
\texttt{IMAGE x y}
returns the result of applying the function
\texttt{x}
to every member of the set
\texttt{y} and 
\texttt{(\$UNION U)} is 
the function that forms the union of its argument with the set $U$:
thus 
\texttt{IMAGE (\$UNION U) st} encodes
$\{U \cup t \mid t \in st\}$.

The type given for \texttt{is\_tab\_rule} uses the type abbreviations
given above for \texttt{rule\_sk} and \texttt{rule} which means that 
the type \texttt{'sf rule\_sk} is an abbreviation for
\texttt{'sf \# 'sf set set}, and
\texttt{'sf set rule} is an abbreviation for \texttt{'sf set \# 'sf
  set set}.

So where a rule-skeleton,
% \marginpar{changed wording of this para also}
as in~\cite[Fig.\ 1.3]{fitting-dual-tableau}, has, for its premise,
a signed formula and, for its conclusion, a set of sets of signed formulae
(so of type \texttt{'sf rule\_sk}),
adding context gives a rule as
in~\cite[Fig.\ 1.6]{fitting-dual-tableau},
which has, for its premise, a set of signed formulae and,
for its conclusion, a set of sets of signed formulae
(so of type \texttt{'sf set rule}).

The definition of \texttt{is\_tab\_rule} is an inductive definition,
which means that \texttt{is\_tab\_rule} is defined to be the predicate which
is satisfied (only) by pairs that can be inferred to satisfy it
using the definition clause given.

For rules such as the last two of~\cite[Fig.\ 1.6]{fitting-dual-tableau},
where only $F$-signed context items are allowed in the
result, we have similar relations:
\begin{definition}
  For all $s$, $st$ and $U$,
  \texttt{is\_ag\_tab\_rule} $(s, st)$
  returns the pair
  \[
    (\{s\} \cup U, \;\{t \cup U_F \mid t \in st \})
    \mbox{ where }  U_F =  \{(F, g) \mid (F, g) \in U\}
  \]
\begin{verbatim}
is_ag_tab_rule : 'a sf rule_sk -> 'a sf set rule -> bool ;
!s st U. is_ag_tab_rule (s, st)
         ({s} UNION U, IMAGE ($UNION (U INTER ($~ o FST))) st)
\end{verbatim}
\end{definition}

Here, the construct
\verb!~! is boolean negation, meaning that it returns true iff its
argument is of type \texttt{bool} and is \texttt{F}.
The construct
\texttt{o} is relational composition
and \texttt{FST} is the function that returns the first
component of a pair. So the construct
\verb!($~ o FST)!
is effectively the set of all
\texttt{F}-signed formulae.
The construct \texttt{INTER} is set intersection so
\texttt{(U INTER (\$\~\ o FST))}
is the set of $F$-signed formulae from
\texttt{U}.
Thus
\texttt{is\_ag\_tab\_rule} allows any context in the premise, but only
$F$-signed context in the conclusion (as in the last two rules of~\cite[Figure~1.6]{fitting-dual-tableau}),

\begin{definition}
For all $s$, $st$ and $U$,
  \texttt{is\_aa\_tab\_rule} $(s, st)$
  returns the pair
  \[
    (\{s\} \cup U_F, \;\{t \cup U_F \mid t\in st\})
    \mbox{ where }  U_F =  \{(F, g) \mid (F, g) \in U\}
  \]
\begin{verbatim}
is_aa_tab_rule : 'a sf rule_sk -> 'a sf set rule -> bool ;
!s st U. is_aa_tab_rule (s, st)
         ( {s} UNION (U INTER ($~ o FST)) , 
           IMAGE ($UNION (U INTER ($~ o FST))) st )
\end{verbatim}
\end{definition}
% rules rewritten to avoid use of INSERT 
% ONCE_REWRITE_RULE [INSERT_SING_UNION] is_ag_tab_rule_rules ;
% ONCE_REWRITE_RULE [INSERT_SING_UNION] is_aa_tab_rule_rules ;

Thus
\texttt{is\_aa\_tab\_rule} allows $F$-signed contexts only, in the
premise and conclusion (this is useful for a lemma we need).

Each rule is defined in its skeletal form (without the
context). For example, the rule-skeleton for $F~ X \lor Y$ shown above
is encoded as below resulting in the type shown:
\begin{verbatim}
!X Y. or_left (
  (F, Or X Y), 
  { {(F, Or X Y); (F, X)} ; {(F, Or X Y); (F, Y)} }
  )
or_left : 'a sf rule_sk set
or_left : ((bool # 'a formula) # ((bool # 'a formula) set set)) set
\end{verbatim}

We then collect these rule skeletons into two sets as below.
\begin{definition}[\texttt{gen\_idt\_rule}, \texttt{ant\_idt\_rule}]
 The following sets of rules are defined in skeleton form:
 \begin{description}
 \item[\texttt{gen\_idt\_rule}:] skeleton form
  of the six rules which allow arbitrary contexts;
\item[\texttt{ant\_idt\_rule}:] the skeletons of the
  \texttt{imp\_right} and \texttt{not\_right} rules which have only a
   $F$-signed context in the result
   (see~\cite[Figure~1.6]{fitting-dual-tableau}).
 \end{description}
\end{definition}

We then define all the rules of the system by taking these skeletons
and allowing contexts appropriately as below.
\begin{definition}[\texttt{idt\_tab\_rule}]\label{idt-tab-rule}
  The set \texttt{idt\_tab\_rule} is inductively defined via the
  two clauses below:
  \begin{description}
  \item[\texttt{gen\_idt\_rule gr}:] For all \texttt{gr}, if
    \texttt{gr} is the skeleton of a rule from the first 6 rules of
    Fig~6, and \texttt{rl} is obtained from \texttt{gr} by adding an
    arbitrary context then \texttt{rl} is a rule of the dual-tableau
    calculus.
  \item[\texttt{ant\_idt\_rule}:] For all \texttt{gr}, if
    \texttt{gr} is the skeleton of a rule from the last two rules of
    Fig~6, and \texttt{rl} is obtained from \texttt{gr} by adding an
    arbitrary context to the premise but adding only the
    \texttt{F}-signed part of this context to the conclusions
    then \texttt{rl} is a rule of the dual-tableau
    calculus.
  \end{description}
\begin{verbatim}
   idt_tab_rule : 'a sf set rule set

   (!gr rl. gen_idt_rule gr 
          /\ is_tab_rule gr rl ==> idt_tab_rule rl) 
   /\
   (!gr rl. ant_idt_rule gr 
          /\ is_ag_tab_rule gr rl ==> idt_tab_rule rl);
\end{verbatim}
\end{definition}

Here, \texttt{gr} will be a pair consisting of the skeleton premise
and the skeleton conclusions while \texttt{rl} will be those pairs,
each extended by some appropriate context.
Again, the above is an inductive definition of the rules
\texttt{idt\_tab\_rule}
for
intuitionistic dual-tableaux so these are the only ways to obtain a
legal rule.

Notice that we do not define a closed dual-tableau,
which is dealt with in the next subsection.

\subsection{Branches, dual-tableaux and their fringes}

Each branch of a dual-tableau ends in a leaf which is a set of
signed formulae. So the set of all leaves of a dual-tableau, which we 
will call its ``fringe'', is a set of sets of signed formulae. 
When we apply a rule to one of these leaves, the effect on the fringe
is to replace that single leaf by the set of leaves 
which is the result of the rule.  
\begin{definition}[\texttt{extend\_fringe}]\label{extend-fringe}
  For all \textit{s}, \textit{sfr} and rule sets \textit{rs}, if
  \textit{rs} contains a rule which takes \textit{s} to the set
  \textit{sfr}, and we apply it to a dual-tableau with a fringe consisting
  of the leaf $s$ plus the 
  %s is one leaf, rf is the set of others
  leaf items \textit{rf} of the other branches, in addition to
  $s$, then the result is the new leaves \textit{sfr} arising
  from \textit{s} plus the unchanged leaves \textit{rf} of the other
  branches.
\begin{verbatim}
! rs s sfr. rs (s, sfr) 
          ==> extend_fringe rs ({s} UNION rf, sfr UNION rf) ;
extend_fringe : 'sfs rule set -> ('sfs set # 'sfs set) set ;
\end{verbatim}
% \marginpar{referee would like to use IN, here and elsewhere, 
% but this is different in HOL 
% (ie, type 'a set is s synonym for 'a $\to$ bool, but 
% x IN S is not the same as S x, just provably equivalent)}
% rules rewritten to avoid use of INSERT 
% ONCE_REWRITE_RULE [INSERT_SING_UNION] fringe_rules_rules ;
\end{definition}

So for a rule set \texttt{rs}, the function
\texttt{extend-fringe rs} gives the set of resulting transformations
of the fringe of a dual-tableau obtained by applying one of the rules
$(s, \mathit{sfr})$.
%
Here, we have written 
the type variable as \texttt{'sfs} to suggest that we will
use \texttt{extend-fringe} where \texttt{'sfs} is  the type of sets of signed
formulae. (We will also use the term variable \texttt{sfs} to indicate
a set of signed formulae).
Moreover, we shall instantiate \texttt{rs} as
\texttt{idt\_tab\_rule} giving the type:
\begin{verbatim}
extend_fringe idt_tab_rule : ('a sf set set # 'a sf set set) set ; 
\end{verbatim}

At this point we note that in \hol, sets and predicates are identified
and so $x \in P$, (\ie \texttt{x IN P}), means exactly $P\
x$ (\ie \texttt{P x}).
Consequently, 
Definition~\ref{extend-fringe} of \texttt{extend\_fringe} might be more clearly written:
\begin{verbatim}
(s, sfr) IN rs 
           ==> ({s} UNION rf, sfr UNION rf) IN extend_fringe rs
\end{verbatim}
This is an inductive definition which means that \texttt{extend\_fringe rs}
is defined to be the set of those fringe-transformations which can
be inferred to be in that set by application of this definition.

The intuition is that we do not keep track of the internal nodes of a
dual-tableau: we keep track of its root (a set of signed-formulae) and
its fringe (a set of sets of signed-formulae).

\subsection{Closed dual-tableaux and a statement of soundness}

By definition, a branch tip, or leaf, is then just a member of the fringe of a
dual-tableau, that is, a leaf is a set of signed formulae. 

\begin{definition}[closed branch]
\label{br-closed}\label{dt-closed}\label{at-closed}
  A branch tip, \ie a leaf, $\mathtt{sfs}$ is closed if it contains some
  formula $\mathtt{f}$ signed
  $\mathtt{F}$ and $\mathtt{T}$:
\begin{verbatim}
   br_closed : 'a sf set -> bool
   br_closed sfs = ?f. (T, f) IN sfs /\ (F, f) IN sfs 
\end{verbatim}

  A dual-tableau (fringe) $\mathtt{sfss}$ is closed if every 
  branch $\mathtt{sfs}$ in it is closed: 
\begin{verbatim}
   dt_closed : 'a sf set set -> bool
   dt_closed sfss = !sfs. sfs IN sfss ==> br_closed sfs 
\end{verbatim}

  A leaf $\mathtt{sfs}$ is atomically closed if it contains some atomic
  formula $\mathtt{Atom\ p}$ signed
  $\mathtt{F}$ and $\mathtt{T}$:
\begin{verbatim}
   at_closed : 'a sf set -> bool
   at_closed sfs = ?p. (T, Atom p) IN sfs /\ (F, Atom p) IN sfs
\end{verbatim}

\end{definition}

% FOLLOWING PARA CHANGED Fri Feb  3 10:40:03 AEDT 2017
Note that we work with closure defined using $T$- and $F$-signed occurrences of
an arbitrary formula $f$ rather than an atomic formula \texttt{Atom} $p$.
Later on (see Lemma~\ref{atomic-closure}) we show
that everything still goes through if we demand that $f$ is atomic.

Now, the action of repeatedly applying dual-tableau rules from some set of
rules can be expressed as the reflexive transitive closure of
application of any rule from that set.
In \hol,, a reflexive transitive closure function \texttt{RTC} is provided:
it takes and returns relations of the type \texttt{'a -> 'a -> bool}.
For a relation $R$ of this type, $aRb$ is expressed in \hol{} as $R\ a\ b$.
%

Thus our soundness theorem will be of the form below:
\begin{quote}
  If $(R, pv)$ is an intuitionistic Kripke model and if the
  dual-tableau fringe $bot$ is obtained from repeated
  applications of the set \texttt{idt\_tab\_rule} of rules to
  the initial fringe
  $\{\{(T,f)\}\}$, \underline{and} $bot$ is closed then the formula
  $f$ is true in every world $w$ of $(R, pv)$:
\begin{verbatim}
Kripke_model R pv ==>
   RTC (CURRY (extend_fringe idt_tab_rule)) {{(T,f)}} bot ==>
   dt_closed bot ==> forces R pv w f
\end{verbatim}
\end{quote}
Here, \texttt{CURRY} : $(\alpha ~\times ~\beta \to bool) \to
\alpha \to \beta \to bool$
takes a relation 
in the form of a predicate of type
$(\alpha ~\times ~\beta \to bool)$
on pairs
$(\alpha,\beta)$,
and returns a relation in the form of the same predicate 
with the type $(\alpha ~\to ~\beta \to bool)$ on two curried arguments
$\alpha$ and $\beta$: which is the form required by 
\texttt{RTC}.
Also, note that the construct
\texttt{A ==> B ==> C} in \hol{} is logical equivalent to the construct
\verb!A /\ B ==> C!, which is why the English prose uses
``\underline{and}'' rather than ``implies''.

Notice that we defined a (portion of a) dual-tableau using reflexive
transitive closure of the relation which takes one fringe to the next,
so the initial fringe is a singleton set 
$\{ \{(T, f)\} \}$
containing the initial leaf
$\{(T, f)\}$.

Note: some of the rules copy their principal formulae into all
of their conclusions. So why do we not have to worry about termination
in the soundness proof? Because, by definition, the
reflexive-transitive closure is obtained by a finite number of
applications, thus each dual-tableau is finite by definition. 

To complete this theorem, we now have to formalise the Kripke
semantics of intuitionistic logic, thereby formalising the notions of 
\texttt{Kripke\_model R pv} 
and 
\texttt{forces R pv w f}.

\section{Formalised  intuitionistic Kripke models}

The Kripke semantics of intuitionistic logic are based upon
classical logic, so we can encode these semantics directly into the
classical higher-order logic of \hol.

Using \texttt{R} to encode the underlying binary relation, using
\texttt{v} and \texttt{w} for worlds, and using \texttt{pv} for the
(classical) propositional valuation of an atom \texttt{a} to one of true or
false at a world, we define a persistent valuation function
directly:
\begin{definition}
 % we use R and pv of curried types
 A binary relation $R$ over some given set of worlds of type
 \texttt{'w} is a function that maps a pair $w$ and $v$ of worlds
 of type \texttt{'w} to \texttt{T} or \texttt{F} depending on
 whether the pair $w,v$ is or is not in the relation,
 ie, whether $wRv$ or not.

 A propositional valuation $pv$ maps a 
 world $w$ of type \texttt{'w} and an atom of type \texttt{'a formula}
 to 
\texttt{T} or else \texttt{F} depending on
 whether the atom $a$ is true or false at world $w$.
\begin{verbatim}
  R  : 'w -> 'w -> bool
  pv : 'w -> ('a formula) -> bool
\end{verbatim}
\end{definition}

\begin{definition}[\texttt{persistent R pv}]
 The classical valuation \texttt{pv} is persistent over a binary
 relation \texttt{R} over some set of worlds if for all worlds
 \texttt{v} and \texttt{w}, and all atoms \texttt{a}, 
 if \texttt{w} is an \texttt{R}-successor of \texttt{v} 
 then if \texttt{a} is true at \texttt{v} then 
 \texttt{a} is true at \texttt{w}:
\begin{verbatim}
   persistent R pv = !v w a. R v w ==> pv v a ==> pv w a
\end{verbatim}
\end{definition}

Using the predicates \texttt{transitive} and \texttt{reflexive} which
are pre-defined in \hol, we define \texttt{R, pv} to be a Kripke model
as follows.
\begin{definition}[Kripke\_model]
\texttt{R} and \texttt{pv} is a Kripke model iff 
the binary relation \texttt{R} is reflexive and
transitive, and the valuation \texttt{pv} is persistent over
\texttt{R}:
\begin{verbatim}
   Kripke_model R pv 
            = transitive R /\ reflexive R /\ persistent R pv 
\end{verbatim}
\end{definition}

Note, currently there is no condition that the set of worlds is
non-empty as is usual in Kripke semantics. For the moment, we do not
need it. As we shall see, it will become essential later in the
completeness proof. Also, note that intuitionistic Kripke frames are
often defined to be reflexive, transitive and anti-symmetric: 
$\forall x, y. R~ x~ y ~\&~ R~ y~ x \Longrightarrow x = y$. 
The ``persistence'' of the binary relation $R$ ensures that the two
definitions give rise to the same notion of validity. But again, we
do not require this extra condition.

\begin{definition}[\texttt{forces R pv w f}]\label{defn-forces}
  The usual forcing relation \texttt{forces R pv w f} that
  holds between a model \texttt{R, pv}, a world \texttt{w} and a
  formula \texttt{f} is then as defined below:
\begin{verbatim}
   (forces R pv w (Atom a)  = pv w a) 
/\ (forces R pv w (And p q) = forces R pv w p /\ forces R pv w q) 
/\ (forces R pv w (Or p q)  = forces R pv w p \/ forces R pv w q) 
/\ (forces R pv w (Not p)   = !v. R w v ==> ~ forces R pv v p) 
/\ (forces R pv w (Imp p q) =
               !v. R w v ==> forces R pv v p ==> forces R pv v q) 

forces : ('w -> 'w -> bool) -> ('w -> 'a -> bool) 
                            -> 'w -> 'a formula -> bool ;
\end{verbatim}
\end{definition}

We say a world $v$ is a \emph{future} world of world $w$ if $R\ w\ v$.

\begin{lemma}[\texttt{FORCES\_PERSISTENT}]
If the binary relation \texttt{R} is transitive and 
the valuation \texttt{pv} of atomic formulae is persistent over \texttt{R} 
then so is the forcing predicate \texttt{forces R pv}:
\begin{verbatim}
  transitive R ==> persistent R pv ==> persistent R (forces R pv)
\end{verbatim}
\end{lemma}

In the above, the two uses of \texttt{persistent} have
different types, the first is about a valuation of atoms while the
second is about the forcing predicate (which is a derived valuation of
formulae).

We obtain an equivalent version of 
\texttt{FORCES\_PERSISTENT}:
\begin{lemma}[\texttt{FORCES\_IF\_ALL}]
If \texttt{R, pv} is a Kripke model then
a world \texttt{w} in the model forces a formula \texttt{f}
if and only if every future world \texttt{v} forces \texttt{f}.
\begin{verbatim}
  Kripke_model R pv ==> 
        ((!v. R w v ==> forces R pv v f) = forces R pv w f)
\end{verbatim}
\end{lemma}

\section{Attributed formulae and soundness}

% The proof of soundness involves attributing a valuation to a fringe
% of\marginpar{falsifiable how?}  the tableau, and that the
% rules preserve falsifiability.

The proof of soundness involves attributing an intuitionistic formula
to each signed-formula set in a fringe of the dual-tableau, and proving
that the rules preserve intuitionistic validity of these attributed
formulae upwards: that is, for each rule, if each intuitionistic
formula attributed to a conclusion of that rule is intuitionistically
valid then so is the intuitionistic formula attributed to the premise
of that rule.

We first tried to encode this notion of a valuation for the
attributed formula directly but got
stuck when, contrary to our expectations, we found that
Lemma~\ref{is-tab-rule-pres-eqv} does not hold for these
valuations. We therefore reworked all the definitions as shown next.

% \marginpar{JED? Answered in footnote a few pages on}
% \textbf{Section 5: You mention that with a certain definition, Lemma 7 does not hold.
% Does this mean that there is a mistake in the Fitting’s proof? If so, I would
% welcome a slightly longer discussion of that: Is that misake severe? Does it
% show that mechanized proofs are important, or does not not really matter.
% If there is no mistake in paper proofs, then I wonder why one should believe or
% care whether Lemma 7 should hold for the auxillary definition.
% }

Given a set \textit{sfs} of signed formulae, the intuitionistic
formula attributed to \textit{sfs} is
$\bigwedge Fs \supset \bigvee Ts$ where \textit{Fs} and \textit{Ts}
are each the set of unsigned formulae that are $F$-signed and
$T$-signed in \textit{sfs}, respectively.  Here, 
the empty disjunction is read as contradiction $\bot$ and 
the implication $p \supset \bot$ is intuitionistically equivalent to
the negation of $p$.
According to
Definition~\ref{defn-forces}, the intuitionistic semantics of
$p \supset q$ (\texttt{Imp p q}) at a world $w$
involves evaluating the classical
logic implication \texttt{forces R pv v p ==> forces R pv v q} over
all $R$-successors
\texttt{v}, so we first encode this ``classical'' notion and put
$Fs$ for $p$ and $Ts$ for $q$.

% Thus we define a ``single-world''
% \marginpar{reconsider word ``single-world''}
% valuation of $(\mathit{Fs}, \mathit{Ts})$
% at a given world \textit{v} of a given Kripke model \textit{R, pv}
% depending upon whether or not the attributed formula (sequent)
% $\bigwedge \mathit{Fs} \supset \bigvee \mathit{Ts}$  is true
% classically at \textit{v}:
\begin{definition}
The predicate \texttt{sfs\_val\_aux R pv v sfs} is true iff: if every
formula \textit{f} signed $F$ in the set \textit{sfs} of signed-formulae
is forced at \textit{v} then some formula \textit{t} signed $T$ in the
set \textit{sfs} of signed-formulae is also forced at \textit{v}. 
\begin{verbatim}
sfs_val_aux R pv v sfs =
  let (Fs, Ts) = mk_seq sfs in 
  (!f. f IN Fs ==> forces R pv v f) 
                              ==> (?t. t IN Ts /\ forces R pv v t)
\end{verbatim}
\end{definition}

Here, we first convert the set \texttt{sfs} of signed formulae into a
sequent $Fs \vdash Ts$ using our previously defined function
\texttt{mk\_seq}.  Then we encode the classical logic formula
$\forall f \in \mathit{Fs}.\ \mathtt{forces}~ R~ pv~ v~ f \Rightarrow \exists t
\in \mathit{Ts}.\ \mathtt{forces} ~R ~pv ~v ~t$ rather than the intuitionistic
formula $\bigwedge \mathit{Fs} \supset \bigvee \mathit{Ts}$ attributed
to \textit{sfs}. To obtain the valuation of the attributed formulae,
we have to evaluate this auxiliary classical formula over all future
worlds.
\begin{definition}
The predicate
\texttt{sfs\_val R pv w sfs}
holds iff every future world 
\texttt{v} of 
\texttt{w} satisfies
\texttt{sfs\_val\_aux R pv v sfs}:
\begin{verbatim}
   sfs_val R pv w sfs = !v. R w v ==> sfs_val_aux R pv v sfs 
\end{verbatim}
\end{definition}

% Note however that what we call the ``single-world'' valuation, 
% \texttt{sfs\_val\_aux}, does in fact look at successor worlds in 
% evaluating a single formula $X \supset Y$ or $\lnot X$.

\begin{definition}
The valuation of the conclusions $\mathit{fss}$ of a branching rule is the
conjunction of the valuations of the signed formula sets $\mathit{sfs}$ in
each conclusion. The valuation of a
dual-tableau fringe \textit{fss} is the conjunction of the valuations
of each constituent leaf \textit{sfs}. Both notions can be
captured by instantiating the definition below appropriately:
\begin{verbatim}
   tab_val R pv w fss = (!sfs. sfs IN fss ==> sfs_val R pv w sfs)
\end{verbatim}
\end{definition}

Again, it is also useful to define a corresponding auxiliary function
giving the ``classical'' valuation of the whole dual-tableau fringe,
or of the conclusions of a rule, at a particular world \textit{v}:
\begin{definition}
The predicate
\texttt{tab\_val\_aux R pv v fss}
holds of a fringe 
\texttt{fss}
iff every leaf
\texttt{sfs}
in the fringe satisfies
\texttt{sfs\_val\_aux R pv v sfs}:
\begin{verbatim}
   tab_val_aux R pv v fss 
        = (!sfs. sfs IN fss ==> sfs_val_aux R pv v sfs)
\end{verbatim}
\end{definition}

The open loop below captures that \texttt{tab\_val\_aux} is defined in
terms of \texttt{sfs\_val\_aux} which is used to define
\texttt{sfs\_val} which is used to define \texttt{tab\_val}:
\begin{verbatim}
           sfs_val_aux --- sfs_val
                |             |
           tab_val_aux      tab_val
\end{verbatim}
The next lemma ``closes the loop'' by expressing \texttt{tab\_val} in terms of
\texttt{tab\_val\_aux}, stating that \texttt{tab\_val} does indeed
evaluate \texttt{tab\_val\_aux} over all future worlds.

\begin{lemma}[\texttt{tab\_val\_alt}]
The predicate \texttt{tab\_val R pv w fss} holds iff 
the auxiliary predicate \texttt{tab\_val\_aux R pv v fss} holds at
every future world \textit{v} of \textit{w}:
\begin{verbatim}
   tab_val R pv w fss = !v. R w v ==> tab_val_aux R pv v fss
\end{verbatim}
\end{lemma}

We want to prove soundness in terms of closed dual-tableaux, so we have

\begin{lemma}[\texttt{idt\_br\_sound},\texttt{idt\_dt\_sound}]
\label{idt-br-sound} \label{idt-dt-sound}
  If a dual-tableau branch is closed, then the auxiliary valuation (using
  \texttt{sfs\_val}) of the leaf of that branch is true, and if a
  dual-tableau is closed, then the valuation (using
  \texttt{tab\_val}) of the fringe is true.
\end{lemma}
\begin{verbatim}
   idt_br_sound : br_closed br ==> sfs_val R pv w br
   idt_dt_sound : dt_closed tab ==> tab_val R pv w tab
\end{verbatim}

The intuition of the above lemma is, of course, that the leaf of
each closed branch contains at least one formula \textit{f} that
appears in both \textit{Fs} and \textit{Ts}, and hence is the witness
for the right-hand side of $\bigwedge \mathit{Fs} \supset \bigvee \mathit{Ts}$.

For the first six rules of \cite[Figure~1.6]{fitting-dual-tableau}
(without the context $S$)
the preservation of validity is in fact an equivalence where we use
\textit{b} as a place-holder for a sign:

\begin{lemma}[\texttt{idt\_rules\_aux\_eqv}] \label{idt-rules-aux-equiv}
  For the skeletons of the first six rules of 
  \cite[Figure~1.6]{fitting-dual-tableau} (without the context $S$),
  the auxiliary valuation of the signed formula
  $(b, f)$ from the rule premise equals the auxiliary
  valuation of the set  \textit{sfss} of signed formula sets from the
  conclusions of the rule, as long as $R$ is reflexive:
\end{lemma}
%idt_rules_aux_eqv : 
\begin{verbatim}
  reflexive R ==> gen_idt_rule ((b,f),sfss) 
      ==> (tab_val_aux R pv w sfss = sfs_val_aux R pv w {(b,f)})
\end{verbatim}

Here, we have deliberately used \texttt{(b, f)} rather than the
equivalent \texttt{sf} to highlight the following: why do we suddenly
need reflexivity?  
Because for the $\supset$-F
rule (ignoring the $S$), where $b=F$ and $f = (X \supset Y)$, the
\texttt{tab\_val\_aux} of the conclusions $F (X \supset Y), F Y$
and $F (X \supset Y), T X$ is the conjunction of the respective
semantic clauses $w \Vdash X \supset Y \Rightarrow w \not\Vdash Y$ and
$w \Vdash X \supset Y \Rightarrow w \Vdash X$ while the
\texttt{sfs\_val\_aux} of the premise $F (X \supset Y)$ is
% CHANGED Fri Feb  3 12:05:00 AEDT 2017
$w \not\Vdash X \supset Y$.  For the former to imply the latter we
require $w \not\Vdash Y$ and $w \Vdash X$ to imply
$w \not\Vdash X \supset Y$.  But $w \not\Vdash X \supset Y$ is
$\exists v.\ w R v \;\&\; v \Vdash X \;\&\; v \not\Vdash Y$, which
holds if we choose $v=w$ by reflexivity of $R$.

Adding the context preserves this property:
\begin{lemma}[\texttt{is\_tab\_rule\_pres\_eqv}] \label{is-tab-rule-pres-eqv}
If a dual-tableau rule \texttt{(sfs, sfss)} 
preserves auxiliary valuations,
then the extension \texttt{(esfs, esfss)} of that rule 
by a context also preserves them.
\end{lemma}
%is_tab_rule_pres_eqv :
\begin{verbatim}
  is_tab_rule (sf, sfss) (esf, esfss) ==>
    (tab_val_aux R pv w sfss = sfs_val_aux R pv w {sf}) ==>
      (tab_val_aux R pv w esfss = sfs_val_aux R pv w esf)
\end{verbatim}

Here, notice that we first need to turn the single signed-formula $sf$
into a set $\{sf\}$ of signed-formulae while $esf$ is a set of
signed-formulae since it is an extension of $sf$ by adding context.

We tried to prove Lemma~\ref{is-tab-rule-pres-eqv}
for the actual (``non-auxiliary'') valuations,
but it doesn't hold.\footnote{
The predicates \texttt{tab\_val\_aux R pv \textit{x} sfss} and 
\texttt{sfs\_val\_aux R pv \textit{y} \{sf\}} 
may be false only at worlds $x = \texttt{u}$ and $y = \texttt{v}$ 
respectively, where \texttt{u} and \texttt{v}
are different future worlds of \texttt{w},
which make
\texttt{tab\_val\_aux R pv w sfss} and \texttt{sfs\_val\_aux R pv w \{sf\}}
equal (both false);
however adding context may change the valuation to make it true
at world \texttt{u} but not \texttt{v}, or vice versa,
which would make 
\texttt{tab\_val\_aux R pv w esfss} and \texttt{sfs\_val\_aux R pv w esf}
unequal.
This doesn't suggest a flaw in Fitting's proof, rather that the level of detail
he gives doesn't indicate precisely the sequence of lemmas to be used.}
This means that the proof of Lemma~\ref{idt-pres-frg}, 
so far as it concerns these rules, depends on first applying 
Lemma~\ref{is-tab-rule-pres-eqv} to Lemma~\ref{idt-rules-aux-equiv}
and only then quantifying over future worlds.  

However Lemma~\ref{idt-rules-aux-equiv} clearly extends to the
actual valuations, and we get a similar equivalence
for the last two rules of~\cite[Figure~1.6]{fitting-dual-tableau}.

\begin{lemma}[\texttt{ant\_rules\_eqv}] \label{ant-rules-eqv} For the
  skeletons of the last two rules of \cite[Figure~1.6]{fitting-dual-tableau}
  (without the context $S$ or $S_F$),
  the valuation of the premise signed formula equals the
  valuation of the set of signed formula sets in the conclusions,
  as long as the relation is reflexive
  and transitive:
\end{lemma}
%ant_rules_eqv :
\begin{verbatim}
 transitive R ==> reflexive R ==> 
    ant_idt_rule ((b, f), sfss) ==>
        (tab_val R pv w sfss = sfs_val R pv w {(b,f)})
\end{verbatim}

Note that we now need both reflexivity and transitivity.

We can characterise the effect of adding antecedent context, that is,
adding $F$-signed formulae to the context:

\begin{lemma}[\texttt{ant\_ctxt\_eqv}] \label{ant-ctxt-eqv}
  For a set $U$ of signed formulae, if
  we add the set $U_F$ of all $F$-signed formulae from $U$
  to a signed formula set $\textit{sfs}$
  then the valuation of the augmented set
  $U_F \cup \textit{sfs}$ is given by:
  $w \Vdash U_F \cup \textit{sfs}$ iff
  forall $v$ such that $wRv$,
  if $v \Vdash (F,f)$ for all $(F,f) \in U_F$
  then $v \Vdash \textit{sfs}$.
\end{lemma}
%ant_ctxt_eqv :
\begin{verbatim}
 Kripke_model R pv ==> 
  (sfs_val R pv w ((U INTER $~ o FST) UNION sfs) =
   !v. R w v ==> 
   (!f. (F, f) IN U ==> forces R pv v f) ==> sfs_val R pv v sfs)
\end{verbatim}

% This useful characterisation\marginpar{I don't understand this.} of
% the semantic effect of adding antecedent context to a signed formula
% set reflects the fact that, in intuitionistic logic, $A \supset (B
% \supset C)$ is $(A \land B) \supset
% C$.  A similar characterisation of the effect of adding succedent
% context is not available because it is not the case that $B \supset (C
% \lor D)$ is expressible (intuitionistically) as $(B \supset C) ~op~
% D$ for any operator $op$.

Here, the right-hand side of the equality 
intuitively captures the valuation of the attributed formula
$U_F \supset (S_F \supset S_T)$ where
\texttt{mk\_seq sfs =} $(S_F, S_T)$
since the outermost quantification over future worlds $v$ captures the
outermost occurrence of $\supset$, and the use of \texttt{sfs\_val}
captures the inner occurrence of $\supset$.
The left-hand side of the equality intuitively captures the valuation
of the attributed formula
$(U_F \land S_F) \supset S_T$
since the inner $\land$ is handled by the
\texttt{UNION} operation and the outer $\supset$ is handled by the
quantification over future worlds inside \texttt{sfs\_val}.
Thus it relies on the intuitionistic logic theorem
$((A \land B) \to C) \leftrightarrow (A \to B \to C)$.
A similar characterisation of the effect of adding succedent
context is not available because it is not the case that $S_F \supset (S_T
\lor U_T)$ is expressible (intuitionistically) as $(S_F \supset S_T) ~op~
U_T$ for any operator $op$.

So we get the following result, that if a dual-tableau rule preserves the
valuation (at all future worlds), then the extension of that rule by
adding antecedent context preserves the valuation (at the present
world).

\begin{lemma}[\texttt{is\_aa\_tab\_rule\_pres\_eqv}] 
  \label{is-aa-tab-rule-pres-eqv}
  If, at all future worlds, the valuation of the conclusions
  $\mathit{sfss}$ of a rule
  equals the valuation of the premise $\mathit{sf}$ of the rule,
  then, when the rule is extended with antecedent context,
  the valuation of the  extended conclusions $\mathit{esfss}$
    equals the valuation of the extended premise $\mathit{esf}$.
\end{lemma}
%is_aa_tab_rule_pres_eqv :
\begin{verbatim}
 Kripke_model R pv 
   ==> is_aa_tab_rule (sf, sfss) (esf, esfss) 
   ==> (!v. R w v ==> (tab_val R pv v sfss = sfs_val R pv v {sf}))
   ==> (tab_val R pv w esfss = sfs_val R pv w esf)
\end{verbatim}
Why is it $\{sf\}$ but not $\{esf\}$?
Because $sf$ is a single signed formula and \texttt{sfs\_val} requires
a set of signed-formulae, while $efs$ is a set of signed-formulae since
it is $\{sf\}$ extended by adding context.

To get from this to the case where the premise can have an arbitrary context,
we just need weakening as shown next.

\begin{lemma}[\texttt{sfs\_val\_wk\_sub}] \label{sfs-val-wk-sub}
If a signed formula set $A$ has valuation true,
then so does any signed formula superset $C$ of $A$.
\end{lemma}
%sfs_val_wk_sub : 
\begin{verbatim}
      A SUBSET C ==> sfs_val R pv v A ==> sfs_val R pv v C
\end{verbatim}

Combining all these results we get the ``upward'' preservation of
valuations from the conclusions of a rule to its premise that we seek.
\begin{lemma}[\texttt{idt\_pres}] \label{idt-pres}
  If we apply a rule to a dual-tableau branch leaf $\mathit{sfs}$,
  and the resulting conclusions $\mathit{sfss}$
  have valuation true,
  then the branch leaf $\mathit{sfs}$ has valuation true.
\end{lemma}
%idt_pres : 
\begin{verbatim}
 Kripke_model R pv 
     ==> idt_tab_rule (sfs, sfss) 
     ==> tab_val R pv w sfss 
     ==> sfs_val R pv w sfs
\end{verbatim}

Now we get the corresponding result for the application of a rule to 
the fringe of a dual-tableau, rather than a single leaf (set of signed formulae).
% \begin{lemma}[\texttt{idt\_pres\_frg}] \label{idt-pres-frg}
% If we apply a rule to a dual-tableau, so that the resulting fringe 
% has valuation true, then the previous fringe has valuation true.
% \end{lemma}
% \begin{verbatim}
% idt_pres_frg :
%   Kripke_model R pv ==> extend_fringe idt_tab_rule (prev, next) 
%                ==> tab_val R pv w next ==> tab_val R pv w prev
% \end{verbatim}
\begin{lemma}[\texttt{idt\_pres\_frg}] \label{idt-pres-frg} If we
  apply a rule to a dual-tableau fringe \texttt{prev}, and the
  resulting fringe \texttt{next} has valuation true, then so does 
\texttt{prev}:
\end{lemma}
%idt_pres_frg :
\begin{verbatim}
  Kripke_model R pv  ==> 
    extend_fringe idt_tab_rule (prev, next) ==> 
       tab_val R pv w next  ==> tab_val R pv w prev
\end{verbatim}

A similar result also holds for the 
reflexive transitive closure of the set of rules, not just a single rule.
\begin{lemma}[\texttt{idt\_rtc\_pres\_frg}] \label{idt-rtc-pres-frg}
  If we apply a sequence of rules to an initial dual-tableau fringe
  \textit{top}, and the resulting fringe \textit{bot} has
  valuation true, then so does the starting fringe \textit{top}:
\end{lemma}
%idt_rtc_pres_frg
\begin{verbatim}
 Kripke_model R pv ==>
    !bot. RTC (CURRY (extend_fringe idt_tab_rule)) top bot ==> 
        tab_val R pv w bot ==> tab_val R pv w top
\end{verbatim}

For a dual-tableau proof of formula \textit{f},
the starting point (the initial fringe) is $\{\{(T,f)\}\}$ and 
we have the following lemma.
\begin{lemma}[\texttt{tab\_val\_single}] \label{tab-val-single}
% If a dual-tableau fringe 
% $\{ ~\{~ (T, f)~ \}~ \}$
% consists of a single branch 
% $\{ ~(T, f)~ \}$
% containing the
% single signed formula $(T, f)$ then the fringe valuation is true
% iff the model forces $f$.
%
The dual-tableau valuation for the fringe $\{ ~\{~ (T, f)~ \}~ \}$ at
a world $w$ of a Kripke model $R, pv$ is
true iff the world $w$ forces $f$.
\end{lemma}
%tab_val_single :
\begin{verbatim}
 Kripke_model R pv ==> 
            (tab_val R pv w {{(T,f)}} = forces R pv w f)
\end{verbatim}

Finally, the soundness result, using 
Lemmas \ref{idt-rtc-pres-frg}, \ref{tab-val-single} and \ref{idt-dt-sound}.
% \texttt{idt\_rtc\_pres\_frg}, \texttt{tab\_val\_single}
% and \texttt{idt\_dt\_sound}.

\begin{theorem}[\texttt{idt\_sound}] \label{idt-sound}
If a dual-tableau for the the signed formula $(T,f)$ is closed
then any model \textit{R, pv} forces \textit{f} at any world
% If a dual-tableau, starting with the signed formula $(T,f)$, is closed,
% then any model \textit{R, pv} forces \textit{f} at any world
\textit{w}.
%idt_sound : 
\begin{verbatim}
 Kripke_model R pv ==>
    RTC (CURRY (extend_fringe idt_tab_rule)) {{(T,f)}} bot 
    ==> dt_closed bot 
    ==> forces R pv w f
\end{verbatim}
\end{theorem}

Here, the dual-tableau starts has an  initial
fringe $\{\{(T,f)\}\}$ and repeatedly applying the dual-tableau rules
to this fringe converts it to the closed fringe $bot$.
Why do we not have explicit universal quantifiers over $R$, $pv$ and
$w$? Because every ``free'' variable in a statement is automatically
considered by \hol{} to be universally quantified.

\section{Formalising completeness} %(file idt\_completeScript.sml)}

We now describe how we formalised Fitting's completeness
proof~\cite[Section~1.3.3]{fitting-dual-tableau} for intuitionistic
dual tableaux.

\subsection{$I$-tautologous sets of signed formulae}

We give two ways to formalise $I$-tautologous sets, one following Fitting
and another using inductive definitions.

\subsubsection{$I$-tautologous sets }

We first define an $I$-tautologous set of signed formulae similarly to
\cite[Definition~7]{fitting-dual-tableau}, but without the
% \marginpar{was Def 4}
requirements that: (i) an $I$-tautologous set S must have a finite
$I$-tautologous subset; and (ii) that the closure of each branch is
atomic; and (iii) that dual-tableaux satisfy the single-use
restriction, whereby only active signed formulae are considered for a
rule application (see \cite[Definitions~1,4]{fitting-dual-tableau}).
% \marginpar{was Def 1}
We also define an $I$-tautologous set of sets of signed formulae (a set
of dual-tableau leaves, \ie a dual-tableau fringe):
\begin{definition}[\texttt{Itautss},\texttt{Itauts}]\label{Itauts-def}
\label{Itauts}
\label{Itautss}
A set \texttt{top} of sets of signed formulae is $I$-tautologous w.r.t.\
a set \texttt{rs} of rules (\ie \texttt{Itautss rs top} holds) if
starting with \texttt{top} and repeatedly applying rules from
\texttt{rs} gives
a fringe \texttt{bot} which is closed. A set \texttt{s} of signed
formulae is $I$-tautologous w.r.t.\ a set \texttt{rs} of rules if
\texttt{Itautss rs \{s\}} holds.
\begin{verbatim}
Itautss : 'a sf set rule set -> 'a sf set set -> bool ;
Itautss rs top = 
  ?bot. RTC (CURRY (extend_fringe rs)) top bot /\ dt_closed bot

Itauts  : 'a sf set rule set -> 'a sf set -> bool ;
Itauts rs s = Itautss rs {s} 
\end{verbatim}
% val Itauts_def = Define `Itauts rs s = Itautss rs {s}` ;
% val Itautss_def = Define `Itautss rs s =
%   ?bot. RTC (CURRY (extend_fringe rs)) s bot /\ dt_closed bot` ;
\end{definition}

We proved
\begin{lemma}[\texttt{ITAUTSS\_ALL}] \label{ITAUTSS-ALL}
A finite set \texttt{sfss} of sets of signed formulae is $I$-tautologous
if and only if each of its member sets \texttt{sfs} is $I$-tautologous.
\begin{verbatim}
  FINITE sfss ==> 
       Itautss rs sfss <=> (!sfs. sfs IN sfss ==> Itauts rs sfs)
\end{verbatim}
\end{lemma}

\begin{lemma}[\texttt{ITAUT\_EX\_RULE}]\label{ITAUT-EX-RULE}
Assuming a set \texttt{rs} of rules is finitely branching,
a set \texttt{top} of signed formulae 
is $I$-tautologous w.r.t.\ \texttt{rs} iff it is closed, 
or there is a dual-tableau rule 
\texttt{(top, rb)}
which can be applied to it and
every resulting branch \texttt{br} in the conclusion 
\texttt{rb} is $I$-tautologous.
%ITAUT_EX_RULE :
\begin{verbatim}
 IMAGE SND rs SUBSET FINITE ==> 
   (Itauts rs top = dt_closed {top} 
    \/ ?rb. (top, rb) IN rs /\ !br. br IN rb ==> Itauts rs br)
\end{verbatim}
\end{lemma}

Here, \texttt{rs} is a set of rules and \texttt{IMAGE SND rs} is the
set of second components (results) of those rules. Thus \texttt{IMAGE
  SND rs} is the set $\{C_1, C_2, \cdots\}$ where
$C_i = \{c_1^i, c_2^i, \cdots \}$ is the set of conclusions of some rule
from \texttt{rs}, where each conclusion $c_j^i$ is a set of signed
formulae. The construct \texttt{FINITE} is the set of all finite sets
so \texttt{X SUBSET FINITE} encodes
$\forall x \in X.\ x \subset \mathtt{FINITE}$.
Thus \texttt{IMAGE SND rs SUBSET FINITE} says that each $C_i$ is
finite, which captures that each rule is finitely branching.

\comment{ deleted - referee's comment
\begin{lemma}
\texttt{IMAGE SND rs SUBSET FINITE} means that
every rule in \texttt{rs} is finitely branching.
\end{lemma}
\begin{verbatim}
IMAGE SND rs SUBSET FINITE <=> !t b. (t, b) IN rs ==> FINITE b
\end{verbatim}
}

\subsubsection{An inductive definition of $I$-tautologous sets}

Lemma~\ref{ITAUT-EX-RULE} seems obvious, but was difficult to prove in
\hol, so we tried to reformulate the definition to make the
mechanics of the \hol{} proofs easier. We therefore defined an $I$-tautologous
set of signed formulae as an inductively defined set, using the fact
stated in Lemma~\ref{ITAUT-EX-RULE}.
% (\texttt{ITAUT\_EX\_RULE})
%as the introduction rule:
\begin{definition}[\texttt{Itauti}]\label{Itauti} 
  For every rule set \texttt{rs}, a set \texttt{top} of signed
  formulae satisfies \texttt{Itauti rs} iff
  \begin{enumerate} 
  \item \label{Itauti-1}
    \texttt{top} is itself closed, \underline{or}
  \item \label{Itauti-2}
   some rule \texttt{(top, rb)} in \texttt{rs} 
   is applicable to \texttt{top} to obtain the conclusion \texttt{rb} 
   and every resulting branch \texttt{br} in \texttt{rb} is
  $I$-tautologous w.r.t.\ \texttt{rs}
  \end{enumerate} 
  and \texttt{Itauti rs} is the unique minimal predicate (set) such that
  \ref{Itauti-1} and \ref{Itauti-2} hold.
\begin{verbatim} 
(!top. br_closed top ==> Itauti rs top) /\ 
(!top. (?rb. (top,rb) IN rs /\ !br. br IN rb ==> Itauti rs br)  
                     ==> Itauti rs top)
\end{verbatim}
\end{definition}

First, note that the linguistic ``\underline{or}'' between clauses~\ref{Itauti-1}
and~\ref{Itauti-2} turns into a logical
``and'' (\verb!/\!) because the English clauses capture the
equivalent definition:
\begin{verbatim} 
!top.( (br_closed top) \/ 
       (?rb. (top,rb) IN rs /\ !br. br IN rb ==> Itauti rs br)
     ) ==> Itauti rs top
\end{verbatim}

However we used the more common and, for proofs, useful,
style of definition, with multiple clauses.
Second, by using  \hol's inductively defined sets, the assertion
contained in the definition (that there is a unique minimal such
predicate) is proved automatically by \hol, as expressed in
Lemma~\ref{Itauti-ind}.

\begin{lemma}[\texttt{Itauti\_ind}]\label{Itauti-ind}
For all rule sets \texttt{rs}
and all predicates \texttt{Itauti'} on signed formula sets, if
\begin{enumerate}
%\item all closed signed formula sets satisfy \texttt{Itauti'}, and
% \item whenever \texttt{(top, rb)} is a rule in \texttt{rs}, and all 
% signed formula sets in \texttt{rb} satisfy \texttt{Itauti'}, 
% then \texttt{top} satisfies \texttt{Itauti'}
\item every closed signed formula set \texttt{top} satisfies
  \texttt{Itauti'}, and
\item whenever \texttt{(top, rb)} is a rule in \texttt{rs}, and every
  signed formula set \texttt{br} in the rule conclusion \texttt{rb} satisfies
  \texttt{Itauti'}, then \texttt{top} satisfies \texttt{Itauti'}
\end{enumerate}
then every signed formula set 
\texttt{a0} satisfying \texttt{Itauti rs} satisfies
\texttt{Itauti'}.
\end{lemma}
\begin{verbatim}
 !rs Itauti'. 
  (!top. br_closed top ==> Itauti' top) /\ 
  (!top. (?rb. (top,rb) IN rs /\ !br. br IN rb ==> Itauti' br) 
         ==> Itauti' top) 
  ==> !a0. Itauti rs a0 ==> Itauti' a0
\end{verbatim}

Intuitively, the lemma states that
any set (predicate) \texttt{Itauti'} which is closed under
the clauses of Definition~\ref{Itauti} for \texttt{Itauti rs} is a superset
of \texttt{Itauti rs}: \ie \texttt{Itauti rs} is the smallest set
satisfying those clauses.

% THIS SEEMS TO BE IN EFFECT A REPETITION OF THE ABOVE 
% The inductive definition of \texttt{Itauti} allows \hol{} to generate the
% following associated induction principle, to prove an arbitrary
% prediate \texttt{P}, which we display as a
% definition:
% \begin{definition}[\texttt{Itauti\_ind}]
% \marginpar{JED: what is this in plain english please ???}
% For all rule sets \texttt{rs} and all properties \texttt{P},
% if the  following two properties hold 
% \begin{enumerate}
% \item \texttt{P top} holds for every \texttt{top} that is closed; and
% \item if \texttt{P top} holds whenever \texttt{P br} holds
%       for every \texttt{top} with a rule \texttt{(top, rb)}
%       in \texttt{rs}
% \end{enumerate}
% then for all \texttt{a}, if \texttt{a} is $I$-tautologous wrt.\
% \texttt{rs} then \texttt{P a} holds.
% \begin{verbatim}
%  !rs P. 
%     (!top. br_closed top ==> P top) 
%  /\ (!top. (?rb.(top,rb) IN rs /\ !br. br IN rb ==> P br) 
%                                                        ==> P top)  
%  ==> !a. Itauti rs a ==> P a
% \end{verbatim}
% \end{definition}

\subsubsection{Relating the two notions of $I$-tautologous sets}
\label{Itauti-Itauts}

We then proved the equivalence of 
Definition~\ref{Itauti} for \texttt{Itauti} and
Definition~\ref{Itauts} for \texttt{Itauts},
under the assumption that rules are finitely branching:
this assumption is required since the definition \texttt{Itauti}
allows the case of an infinitely branching dual-tableau of finite depth.

\begin{lemma}[\texttt{ITAUTS\_EQ\_I}] \label{ITAUTS-EQ-I}
For every rule set \texttt{rs},
if the rules in \texttt{rs} are finitely branching then
the properties \texttt{Itauti} and \texttt{Itauts} are equivalent.
\end{lemma}
%ITAUTS_EQ_I
\begin{verbatim}
 IMAGE SND rs SUBSET FINITE ==> (Itauts rs = Itauti rs)
\end{verbatim}
% Here, \texttt{IMAGE SND rs} returns a set of branches, 
% each branch a set of sets of signed formulae.
  
Note that the equivalence does not hold for (even a finite set of)
infinitely branching rules because an infinitely branching rule can
give an infinite dual-tableau of finite depth, in which each branch is
finite.  If such a dual-tableau is closed then it meets the definition of
\texttt{Itauti}, but not the definition of \texttt{Itauts} since it
does not close in a finite number of rule applications; to put it
another way, definition \texttt{Itauts} involves a finite number of
dual-tableau steps, whereas definition \texttt{Itauti} involves a dual-tableau
of finite depth.

The definition of \texttt{Itauti} was easier to work with than that of
\texttt{Itauts}, avoiding our difficult earlier proof of 
Lemma~\ref{ITAUT-EX-RULE}.
% \marginpar{this para added, can't quantify relative effort as previous proof
% of Lemma~\ref{ITAUT-EX-RULE} has been deleted}
However, using both definitions and proving their equivalence
(in the case of finitely branching rules) essentially shows that,
even when using the definition \texttt{Itauti}, we need only finitely many steps
(which is implicit in the way \texttt{Itauts} is defined).

We then obtained further necessary results, such as the monotonicity
of \texttt{Itauti} and that $I$-tautologous is a property of finite
character (i.e.\ whether a set is $I$-tautologous depends on whether its
finite subsets are):

\begin{lemma}[\texttt{Itauti\_idt\_mono}] \label{Itauti-idt-mono}
  Every superset of an $I$-tautologous set is $I$-tautologous: for every
  \texttt{s}, if \texttt{s} is $I$-tautologous w.r.t.\ the set
  \texttt{idt\_tab\_rule} of dual-tableau rules then so is every
  superset \texttt{t} of \texttt{s}:\marginpar[JED: is this
  correct?]{}
\end{lemma}
%Itauti_idt_mono
\begin{verbatim}
  !s. Itauti idt_tab_rule s 
                  ==> !t. s SUBSET t ==> Itauti idt_tab_rule t
\end{verbatim}

\begin{lemma}[\texttt{ITAUTI\_IDT\_FINITE}]\label{ITAUTI-IDT-FINITE}
Every $I$-tautologous set \texttt{s} has
a finite $I$-tautologous subset \texttt{t}:
\end{lemma}
%ITAUTI_IDT_FINITE
\begin{verbatim}
 !s. Itauti idt_tab_rule s ==>
             ?t. FINITE t /\ t SUBSET s /\ Itauti idt_tab_rule t
\end{verbatim}

%FOLLOWING (to Lemma atomic-closure) NEW/CHANGED Fri Feb  3 11:09:02 AEDT 2017
At this point we recall that a tableau branch leaf is closed if it contains
signed formulae $(F, X)$ and $(T, X)$ for some formula $X$.
A leaf is atomically closed if it contains $(F, p)$ and $(T, p)$
for some atomic formula $p$ (in our encoding, $p = \texttt{Atom}\ a$ for some
atom $a$).
Fitting \cite[Definition~2]{fitting-dual-tableau} uses atomic closure
in his completeness proofs but
we have used closure without this restriction. We 
want to examine whether this makes any real difference.

Having defined \texttt{Itauti} in Definition~\ref{Itauti},
we now define a generalised version \texttt{Itautg},
which is like \texttt{Itauti} except that it allows us
to specify the requirement for a leaf to be considered closed
(thus \texttt{Itauti} = \texttt{Itautg br\_closed}).

\begin{definition}[\texttt{Itautg}]\label{Itautg} 
  For every predicate \texttt{cl} on signed formula sets and
  every rule set \texttt{rs}, a set \texttt{top} of signed
  formulae satisfies \texttt{Itautg cl rs} iff
  \begin{enumerate} 
  \item \label{Itautg-1}
    \texttt{top} satisfies \texttt{cl}, or
  \item \label{Itautg-2}
   some rule \texttt{(top, rb)} in \texttt{rs} 
   is applicable to \texttt{top} to obtain the conclusion \texttt{rb} 
   and every resulting branch \texttt{br} in \texttt{rb} is
  $I$-tautologous w.r.t.\ \texttt{cl} and \texttt{rs}
  \end{enumerate} 
  and \texttt{Itautg cl rs} is the unique minimal predicate (set) such that
  \ref{Itautg-1} and \ref{Itautg-2} hold.
\begin{verbatim} 
(!top. cl top ==> Itautg cl rs top) /\ 
(!top. (?rb. (top,rb) IN rs /\ !br. br IN rb ==> Itautg cl rs br)  
                     ==> Itautg cl rs top)
\end{verbatim}
\end{definition}

Then \texttt{Itautg at\_closed idt\_tab\_rule} is the set of
$I$-tautologous sets, defined in terms of atomic closure, for
intuitionistic dual-tableaux.

We then showed 
that a closed dual-tableau can be extended to an atomically closed
dual-tableau, so
that requiring dual-tableau closure to be atomic makes no difference:
this justifies our approach to simplify proofs by not working throughout
in terms of atomic closure.

\begin{lemma}[\texttt{atomic\_closure}]\label{atomic-closure}
A set is $I$-tautologous (per Definitions \ref{Itauti} and \ref{br-closed})
iff it is $I$-tautologous (defined to require atomic closure):
\end{lemma}
\begin{verbatim}
Itauti idt_tab_rule sfs <=> Itautg at_closed idt_tab_rule sfs 
\end{verbatim}

We now discuss the three assumptions which we did not incorporate
in our definition of $I$-tautologous:
\begin{description}
\item[\rm (i) Finite character:] Fitting defines that an $I$-tautologous
  set $S$ must have a finite $I$-tautologous subset.
  Our Definition \ref{Itauti} does not require this, but we proved,
  in Lemma~\ref{ITAUTI-IDT-FINITE}, that it holds as a consequence.
  Our $I$-tautologous sets are built from dual-tableaux, and each
  such dual-tableau is a finite structure. Thus if the root $\{S\}$ of the
  dual-tableaux contains an infinite set $S$ of signed formulae, then we
  can be assured that our finite dual-tableau will ``touch'' only a
  finite subset of its members. Indeed, this is essentially the reason
  why Lemma~\ref{ITAUTI-IDT-FINITE} holds.
\item[\rm (ii) Atomic closure:] We dropped this assumption and allowed
  closure on arbitrary formulae as it made our task easier.
  As discussed above, we have
  since gone back and proved Lemma~\ref{atomic-closure} that
  everything also goes through if we demand atomic
  closure. Essentially, this required us to prove that a dual-tableau
  which is closed using non-atomic closure can be extended to a
  dual-tableau which is closed atomically.
\item[\rm (iii) Single use restriction:] This restriction is also
  redundant: in fact, by inspection, it can be seen that applying a
  rule to an inactive formula does not make progress towards a closed
  dual-tableau.  This is noted by Fitting when he observes that ``Dual
  tableaus are sound and complete with or without a single use
  restriction, but a single use restriction is better for proof
  search. Indeed, it easily gives us decidability.''  That each
  formula is ``principal'' only once is also redundant as already
  stated by Fitting~\cite[just after his Definition~2]{fitting-dual-tableau}.
%  \marginpar{where ???}
\end{description}

\subsection{The Lindenbaum construction}

We now discuss proving Fitting's ``after
Lindenbaum'' theorem~\cite[Theorem~1]{fitting-dual-tableau}. 
Fitting assumes that the
set of signed formulae is countable.
%
We proved the general lemma which expresses 
the effect obtained from the Lindenbaum construction.

\begin{definition}[\texttt{maxnon}]\label{maxnon}
\texttt{maxnon} $P\ s$ means that the set $s$ does not satisfy the 
predicate $P$, but that every proper superset of $s$ satisfies $P$.
\begin{verbatim}
   maxnon : ('a set -> bool) -> 'a set -> bool
   maxnon_def : maxnon P s = ~ P s /\ !t. s PSUBSET t ==> P t 
\end{verbatim}
\end{definition}

Here, \texttt{PSUBSET} captures $s \subset t$ (the proper subset relation).

\begin{definition}[\texttt{ctns1}]\label{ctns1}
\texttt{ctns1} $cs\ m$ means that the set $m$ contains at least one member
of the set of sets $cs$.
\begin{verbatim}
   ctns1 : ('a set set) -> 'a set -> bool
   ctns1_def : ctns1 cs m <=> ?c. c IN cs /\ c SUBSET m
\end{verbatim}
\end{definition}

\begin{lemma}[\texttt{MAXNON\_CTNS1}] \label{MAXNON-CTNS1} Provided
  that we are dealing with members of a countable set $U$, if
  \texttt{cs} is a set of finite subsets of $U$,
  \texttt{m} $\subseteq U$, and \texttt{m}
  does not contain any member of \texttt{cs},
  then there exists a set \texttt{s} $\subseteq U$ 
  which is a superset of \texttt{m} and does not contain any
  member of \texttt{cs} and is maximal with that property.
\begin{verbatim}
countable (UNIV : 'a set) 
==> (cs : 'a set set) SUBSET FINITE  ==> ~ (ctns1 cs (m : 'a set)) 
==> ?s : 'a set. m SUBSET s /\ maxnon (ctns1 cs) s
\end{verbatim}
\end{lemma}

% \textbf{- p. 19, Lemma 24. The property ctns1 could be explained in the text,
% \marginpar{DONE, see defn above}
% one can infer its meaning from the context but it could be good
% to write explicitly what it means,
% }

Here, we take $U$ to be the set of all
members of its type, so $U$ is \texttt{UNIV}, the set of all things
(of the type in question), and then
\texttt{m} $\subseteq U$, \texttt{s} $\subseteq U$ and,
for $c \in \texttt{cs}$, $c \subseteq U$ hold automatically, which is
why they do not appear explicitly in the encoding but do appear in the
plain text.

From \texttt{ITAUTI\_IDT\_FINITE} and \texttt{MAXNON\_CTNS1} we proved
the following lemma.
\begin{lemma}[\texttt{LINDENBAUM\_I}]\label{LINDENBAUM-I}
  Provided that the set of all signed-formulae is countable, 
  if \texttt{s} is not $I$-tautologous then \texttt{s} has a superset 
\texttt{M} which is maximal non-$I$-tautologous:
\end{lemma}
%LINDENBAUM_I
\begin{verbatim}
 countable (UNIV : 'a sf set) 
  ==> ~ (Itauti idt_tab_rule s) 
  ==> ?M : 'a sf set. s SUBSET M /\ maxnon (Itauti idt_tab_rule) M
\end{verbatim}

% We can then ensure that the set of formulae, and
To use this result, we prove that
the set of signed-formulae, are countable as follows.
\begin{lemma}[\texttt{FORMULAE\_COUNTABLE}, \texttt{SF\_COUNTABLE}]
\label{FORMULAE-COUNTABLE}
\label{SF-COUNTABLE}
If the set \texttt{UNIV : 'a set} of all atoms is countable then the set
\texttt{UNIV : 'a formula set} of all formulae (built from
those atoms) is countable, as is the set
\texttt{'a sf set} of all signed-formulae.
\end{lemma}
%FORMULAE_COUNTABLE
%SF_COUNTABLE
\begin{verbatim}
 countable (UNIV : 'a set) ==> countable (UNIV : 'a formula set)
 countable (UNIV : 'a set) ==> countable (UNIV : 'a sf set)
\end{verbatim}

A simple way to ensure that the set of atomic formulae is countable is
to assume that they are indexed by the natural numbers: for example,
as the infinite set $p_0, p_1, p_2, \cdots$. In \hol{}, we can achieve
our goal by specifying that the type variable \texttt{'a} in the type
\texttt{'a sf} of signed formulae is, in fact, the type \texttt{num}
of natural numbers. 
We thus obtain %thereby getting

\begin{lemma}[\texttt{LINDENBAUM}]\label{LINDENBAUM}
  Assume that the atomic formulae are indexed by the natural numbers:
  that is, let \texttt{'a} be \texttt{num} in \texttt{'a sf}.
  Then, if \texttt{s} is non-$I$-tautologous, then \texttt{s}
  has a superset \texttt{M} which is maximal non-$I$-tautologous.
\end{lemma}
%LINDENBAUM
\begin{verbatim}
  ~ (Itauts idt_tab_rule s) ==>
  ?M : num sf set. s SUBSET M /\ maxnon (Itauts idt_tab_rule) M
\end{verbatim}
Here, we specify the type of \texttt{M} as \texttt{num sf set}
which causes \texttt{'a sf} to be instantiated to \texttt{num sf}.
That is, \texttt{num sf} is \texttt{bool \# num formula}: see
Section~\ref{sec:unsigned-formulae}.

\subsection{The canonical model, Truth Lemma and completeness}

The canonical model is built out of a (non-empty) set of ``worlds''
built from maximal non-$I$-tautologous
sets~\cite[just above Theorem~3]{fitting-dual-tableau}. We therefore define a new type
\texttt{worlds} representing the set of maximal non-$I$-tautologous
sets. But first, we have to show that this set is non-empty, because
types in \hol{} are non-empty.

\begin{lemma}[\texttt{EX\_NON\_TAUT}]\label{EX-NON-TAUT}
If the atomic formulae are indexed by the natural numbers
then there is a maximal non-$I$-tautologous set of signed formulae.
\begin{verbatim}
   ?(M :num sf set). maxnon (Itauts idt_tab_rule) M
\end{verbatim}
\end{lemma}

\begin{definition}
  The new type \texttt{worlds} is isomorphic to the set of 
  maximal-non-$I$-tautologous sets.
\begin{verbatim}
val worlds_TY_DEF = new_type_definition ("worlds", EX_NON_TAUT) ;
\end{verbatim}
\end{definition}

That is, we define the new type \texttt{worlds} to be isomorphic to
the set of things satisfying the property \texttt{maxnon (Itauts
  idt\_tab\_rule)}: namely the set of maximal non-$I$-tautologous sets
which we have just shown to be non-empty by
Lemma~\ref{EX-NON-TAUT}.
%
The function \texttt{new\_type\_definition} also creates functions
and a theorem expressing this isomorphism.

\begin{lemma}[\texttt{worlds\_abs\_rep}]
Assuming the atomic formulae are indexed by the natural numbers,
there exists a function \texttt{w\_rep} from \texttt{worlds} to 
maximal non-$I$-tautologous sets 
and a function \texttt{w\_abs} from 
sets of signed formulae to \texttt{worlds} such that:
\begin{enumerate}
\item for every world \texttt{a} of type \texttt{worlds}, \texttt{w\_abs
    (w\_rep a) = a}; and
\item for every set \texttt{s} of signed formulae 
  \texttt{w\_rep (w\_abs s) = s} iff \texttt{s} is
  maximal non-$I$-tautologous wrt.\ \texttt{rs}.
\end{enumerate}
\begin{verbatim}
w_rep : worlds -> num sf set
w_abs : (num sf set) -> worlds

 ( !(a :worlds). w_abs (w_rep a) = a ) /\ 
    ( !(s :num sf set). (w_rep (w_abs s) = s) <=> 
        maxnon (Itauts (idt_tab_rule :(num sf set) rule set)) s )
\end{verbatim}
\end{lemma}

Here, we specify that the atomic formula are indexed by the natural
numbers by setting the type of $s$ to be \texttt{num sf set}.

We now have a set of worlds built out of maximal non-$I$-tautologous
sets of signed formulae. We define the canonical model over these
worlds by defining the valuation of atoms over these worlds and the
binary Kripke relation between worlds.  

\begin{definition}[\texttt{at\_val, idt\_R}]\label{at-val,idt-R}
  The truth value \texttt{at\_val} of an atomic formula \texttt{Atom
    a} at a world \texttt{w} is true iff \texttt{(F, Atom a)} is in
  the set \texttt{w}.
  The world $\Delta$ is an \texttt{idt\_R}-successor of the world
  $\Gamma$ iff
  $\{f \,|\, (F, f) \in \Gamma\} \subseteq \{f \,|\, (F, f) \in
  \Delta\}$.
\begin{verbatim}
  at_val w a = (F, Atom a) IN w_rep w
  idt_R gamma delta = 
     (FST (mk_seq (w_rep gamma)) SUBSET FST (mk_seq (w_rep delta)))
\end{verbatim}
\end{definition}

Here, the isomorphism function \texttt{wrep} provided by \hol{}
identifies a world \texttt{delta} with its corresponding
set \texttt{w\_rep delta} of signed-formulae, and similarly for world
\texttt{gamma}. We then ``partition'' the $F$-signed formulae from the
$T$-signed formulae from these
sets of signed formulae by turning each into the sequents
$Fs_{\Gamma} \vdash Ts_{\Gamma}$ and
$Fs_{\Delta} \vdash Ts_{\Delta}$, respectively, using \texttt{mk\_seq}.
Projecting onto the first component of these sequents gives us
$Fs_{\Gamma}$ and $Fs_{\Delta}$, respectively,
and the \texttt{SUBSET} construct then gives us the desired result.
%% was  idt_R gamma delta = w_rep gamma SUBSET (FST UNION w_rep delta)
% We then form its union
% with  the set of all pairs whose first component is \texttt{T} via the
% construct \texttt{(FST UNION w\_rep delta)}. Now if 
% \texttt{w\_rep gamma}, the set of signed formulae that 
% represents \texttt{gamma}, is a subset of 
% \texttt{(FST UNION w\_rep delta)}, then all the $F$-signed formulae from 
% \texttt{w\_rep gamma} must already appear in 
% \texttt{ w\_rep delta} since the extra signed-formulae that we added
% to
% \texttt{ w\_rep delta} to
% obtain 
% \texttt{(FST UNION w\_rep delta)} were all signed $T$.

The canonical model is thus built from \texttt{worlds},
\texttt{at\_val} and \texttt{idt\_R} in the usual way and we need to
prove the Truth Lemma.
For proving the Truth Lemma, we proved

\begin{lemma}[\texttt{NON\_ITAUT\_RULE}]\label{NON-ITAUT-RULE}
  If the rules from the rule set \texttt{rs} are finitely branching, 
  and \texttt{s} is maximal non-$I$-tautologous w.r.t.\ \texttt{rs},
  and all extensions by a context of the skeleton rule
  (\texttt{top/bot}) are contained
  in \texttt{rs}, then if \texttt{top} is in \texttt{s}
  then so is some member of \texttt{bot}.
\begin{verbatim}
 IMAGE SND rs SUBSET FINITE 
      ==> maxnon (Itauts rs) s 
      ==> is_tab_rule (top, bot) SUBSET rs 
      ==> top IN s  ==> ?br. br IN bot /\ br SUBSET s
\end{verbatim}
\end{lemma}

Assume the canonical model is built from \texttt{worlds},
\texttt{at\_val} and \texttt{idt\_R} in the usual way using
Definition~\ref{at-val,idt-R}, thus giving rise to a forcing relation
\texttt{forces idt\_R at\_val} which maps a particular 
world $\Gamma$ and a particular formula $X$ to true or false.
%
The following result corresponds to Fitting's 
``Intuitionistic Truth Lemma''~\cite[Theorem~3]{fitting-dual-tableau}.
It is proved by induction on the formula X, using 
% the \texttt{NON\_ITAUT\_RULE} 
Lemma~\ref{NON-ITAUT-RULE}.
\begin{lemma}[\texttt{TRUTH\_LEMMA}]\label{TRUTH-LEMMA}
  For all formulae $X$ and 
  for all worlds $\Gamma$ (ie, maximal non-$I$-tautologous sets of signed
  formulae)
  \begin{enumerate}
  \item if $(T, X)$ in $\Gamma$ then $\Gamma$ does not force $X$, and
  \item if $(F, X)$ in $\Gamma$ then $\Gamma$ does force $X$:
  \end{enumerate}
\begin{verbatim}
 !X gamma. 
  ((T, X) IN w_rep gamma ==> ~ (forces idt_R at_val gamma X)) /\ 
  ((F, X) IN w_rep gamma ==>   (forces idt_R at_val gamma X))
\end{verbatim}
\end{lemma}

Again, we utilise the isomorphism function
\texttt{w\_rep} to find the set of signed formulae represented by $\Gamma$.

For the completeness theorem, we first state a lemma about the
canonical model.

\begin{lemma}[\texttt{idt\_complete}]\label{idt-complete}
In the canonical model, if every world $w$ forces formula $f$ 
then the singleton signed formula set ${\{ (T, f) \}}$ is $I$-tautologous.
\begin{verbatim}
 (!w. forces idt_R at_val w f) ==> Itauts idt_tab_rule {(T,f)}
\end{verbatim}
\end{lemma}

Now, using the contrapositive form, we get completeness as desired:
\begin{quote}
  if no dual-tableau for the set $\{(T, f)\}$ is closed then $f$ is
  falsifiable in some Kripke model~\cite{fitting-dual-tableau}.
\end{quote}

\begin{theorem}[\texttt{idt\_complete\_cp}]\label{idt-complete-cp}
  If the singleton signed formula set ${\{ (T, f) \}}$ is not
  $I$-tautologous (ie. the formula $f$ has no dual-tableau proof),
  then there is a world in the canonical model which does not force $f$.
\begin{verbatim}
 ~ Itauts idt_tab_rule {(T,f)} ==> ?w. ~ forces idt_R at_val w f
\end{verbatim}
\end{theorem}
\begin{proof}
  For a formula $f$, if $\{(T, f)\}$ has no closed dual-tableau, that is,
  if  $\{(T, f)\}$ is not $I$-tautologous, then by Lemma~\ref{LINDENBAUM}, it
  is contained in a maximal non-$I$-tautologous set $\Gamma$, which is a world 
  in the canonical model.  Then, by Lemma~\ref{TRUTH-LEMMA},
  $\Gamma \not\Vdash f$.
\end{proof}


% \subsection{How exactly does the above prove completeness?}

% We first state the notion of a formula being ``provable'' in terms of
% dual-tableaux.
% \begin{definition}
%   An (unsigned) formula $f$ is provable iff there is some fringe
%   \texttt{bot} which is obtained from the initial fringe
%   \texttt{\{\{(T,f)\}\}} by repeatedly applying rules from the rule
%   set \texttt{id\_tab\_rule} and \texttt{bot} is closed.
% \begin{verbatim}
%    ?bot. RTC (CURRY (extend_fringe idt_tab_rule)) {{(T,f)}} bot 
%          /\ dt_closed bot
% \end{verbatim}

%   Equivalently according to Definition~\ref{Itauts-def} of
%   \texttt{Itauts} and \texttt{Itautss}, an (unsigned) formula $f$ is
%   provable iff \texttt{Itauts idt\_tab\_rule {(T,f)}} holds, which is
%   the form used in the \hol{} code from our Theorems~\ref{idt-sound}
%   and \ref{idt-complete}.
% \end{definition}

% To state that \texttt{f} is valid, we need to say that for all Kripke
% models \texttt{R pv}, every world forces \texttt{f}: that is,
% \texttt{!w.\ forces R pv w f}.  Equivalently, for all possible world
% sets and all Kripke models \texttt{R, pv} based on
% those\marginpar{I cannot see this.}
% worlds, every world forces \texttt{f}: that is,
% \texttt{!R pv w.\ Kripke\_model R pv ==> forces R pv w f}.
% But we cannot express the quantification over possible sets of worlds,
% because this would require quantifying over the possible types of worlds, 
% which is not possible in the simply typed higher order logic of HOL.

% Now, in the statement of Theorem~\ref{idt-sound}, the universal quantification
% over the types of worlds is implicitly expressed, since the type appears as a
% free variable in the statement of the theorem.  Then the set of worlds is
% taken to be the set of all members of that type.
% This in fact means that we have only proved
% validity of f in relation to non-empty sets of worlds,
% since types in HOL are non-empty.
% This is hardly a serious limitation, since the Kripke semantics
% demands that the set of worlds is non-empty, but is acknowledged.

% A statement of the completeness theorem runs like this:
% \begin{quote}
% if $f$ is valid in all Kripke models then $f$ is provable,
% that is, if, for all possible sets of worlds,
% \texttt{!R pv w. Kripke\_model R pv ==> forces R pv w f}
% then \texttt{Itauts idt\_tab\_rule \{(T,f)\}}.
% \end{quote}

% We have proved this in the following partly informal way:
% \begin{quote}
%   assuming that $f$ is valid (in all Kripke models), then
%   it is valid in the canonical Kripke model, and, assuming
%   the latter, we have proved formally in Theorem~\ref{idt-complete}
%   that $f$ is provable.
% \end{quote}

% But only in detailing this informal argument did we note that
% we should prove that our model using \texttt{idt\_R} and
% \texttt{at\_val} is in fact a Kripke model:
% this fairly obvious result was proved as
% Lemma~\ref{Kripke-worlds}. % \texttt{Kripke\_worlds}.

% \begin{lemma}[\texttt{Kripke\_worlds}]\label{Kripke-worlds}
%   The canonical model from Definition~\ref{at-val,idt-R} is a Kripke
%   model.
% \begin{verbatim}
%         Kripke_model idt_R at_val
% \end{verbatim}
% \end{lemma}

\subsection{Relaxing the countable constraint}

The proof described above required that the set of formulae is
countable: proving that this holds, if the set of atoms is countable,
was not trivial (see Lemma~\ref{SF-COUNTABLE}).
An alternative is to drop this requirement and to use Zorn's lemma, which is
provided in \hol, giving a version of Lemma~\ref{MAXNON-CTNS1}
without the countable set restriction.

\begin{lemma}[\texttt{MAXNON\_CTNS1\_ZORN}] \label{MAXNON-CTNS1-ZORN}
  If cs is a set of finite sets, and m does not contain any member of cs,
  then there exists an s which is a  superset of m 
  and does not contain any
  member of cs and is maximal w.r.t.\ that property.
%MAXNON_CTNS1_ZORN
\begin{verbatim}
    cs SUBSET FINITE 
       ==> ~ (ctns1 cs m) 
       ==> ?s : 'a set. m SUBSET s /\ maxnon (ctns1 cs) s
\end{verbatim}
\end{lemma}

Both these approaches require the finite character property of 
a set being $I$-tautologous: that is, that an $I$-tautologous set has
an $I$-tautologous finite subset.

Finite characterisation of being $I$-tautologous is conceptually easy,
as discussed earlier, and proved in Lemma~\ref{ITAUTI-IDT-FINITE}.
% since the process of building a closed dual-tableau is finite,
% and each rule application involves only a finite number of formulae
% (other than a context which is unchanged), so only a finite subset of
% the initial set of signed formulae is required. Again, however,
% formalising this requires quite a number of proof steps.
%
However another approach here is to define an $I$-tautologous set as one which
has a finite $I$-tautologous subset, as Fitting does, in
\cite[Definition~7]{fitting-dual-tableau}. 
We did this (calling it \texttt{fITauts}), which made it easy to
prove analogues of the results
% \texttt{Itauti\_idt\_mono} and \texttt{ITAUTI\_IDT\_FINITE} above
Lemmas \ref{Itauti-idt-mono} and \ref{ITAUTI-IDT-FINITE}, but other
things become more difficult.  For example, we proved (at quite some
length) this analogue of % the \texttt{NON\_ITAUT\_RULE}
Lemma~\ref{NON-ITAUT-RULE}.
Note that, compared with Lemma~\ref{NON-ITAUT-RULE},
we proved it only specifically for the set of rules
for intuitionistic dual-tableaux.

\begin{lemma}[\texttt{NON\_FITAUT\_RULE}]

  If $s$ is maximal non-$I$-tautologous wrt.\ the rules \texttt{idt\_tab\_rule}
  for intuitionistic dual-tableaux, 
  and the extensions by a context of the skeleton rule
  \texttt{(top/bot)} are contained
  in \texttt{idt\_tab\_rule}, then if \texttt{top} is in $s$,
  then so is some member of \texttt{bot} 

\begin{verbatim}
   maxnon (fItauts idt_tab_rule) (s : 'a sf set) 
       ==> is_tab_rule (top, bot) SUBSET idt_tab_rule 
       ==> top IN s ==> ?br. br IN bot /\ br SUBSET s
\end{verbatim}
\end{lemma}

We didn't pursue this approach further,
and Lemma~\ref{ITAUTI-IDT-FINITE} 
% and the result \texttt{ITAUTI\_IDT\_FINITE},
makes it rather redundant.
It really just illustrates that until one actually performs the proofs,
one doesn't really know which approach will be simplest to prove.

\section{Conclusions}

We have shown how to encode the meta-theory of dual-tableaux for
intuitionistic logic into \hol. In the process, we have verified all
of the theorems provided by Melvin Fitting in his chapter in this
volume, although our proofs sometimes proceed differently. We have
also highlighted how inductive definitions often make proofs easier
since we can perform structural induction on the clauses that make up
the inductive definition. All of our \hol-code
can be found via the link (\url{http://users.cecs.anu.edu.au/~jeremy/hol/idt/}).

Regarding the effort required. 
The proof script is {2100} lines of \hol-code.
Contrasted against Fitting's original
chapter~\cite{fitting-dual-tableau}, this is a similar length --- 
% \marginpar{blowup factor not meaningful}
but containing much more detail of small proof steps, and much less descriptive
and explanatory material.
This contains some results which were proved in a roundabout way, or with
some duplication of effort (such as the issue of \texttt{Itauti} versus
\texttt{Itauts}, see \S\ref{Itauti-Itauts}),
and a small amount of theory not specific to this particular task,
such as the proof of Lemma~\ref{SF-COUNTABLE}.
(Generally HOL offers good support for most common generic reasoning tasks, 
although not for proving an algebraic data type to be countable).
% \marginpar{Question answered}

One caveat: Jeremy Dawson has over 20 years
of experience in interactive theorem proving, and yet it took him 2
months of full-time work to complete these proofs, so interactive
theorem proving is time-consuming and laborious! 

\paragraph{Acknowledgements.} We are grateful to the anonymous
reviewers for their suggestions for improvements.

% \bibliography{dualtableaux}
 \bibliographystyle{alpha}

\begin{thebibliography}{OGP11}

\bibitem[Bet53]{beth-padoas}
E~W Beth.
\newblock On {Padoa's} method in the theory of definition.
\newblock {\em Indag. Math.}, 15:330--339, 1953.

\bibitem[DG02]{dawson-gore-formalised-cut-admissibility}
Jeremy~E. Dawson and Rajeev Gor{\'e}.
\newblock Formalised cut admissibility for display logic.
\newblock In {\em TPHOLs02:
  Proceedings of the 15th International Conference on Theorem Proving in Higher
  Order Logics}, volume LNCS~2410:131--147. Springer, 2002.

\bibitem[DG10]{DBLP:conf/lpar/DawsonG10} Jeremy~E. Dawson and Rajeev
  Gor{\'{e}}.  \newblock Generic methods for formalising sequent
  calculi applied to provability logic.  \newblock In LPAR-17: {\em
    Proceedings of the International Conference on Logic for
    Programming, Artificial Intelligence, and Reasoning},
  LNCS~6397:263--277, 2010.

\bibitem[Fit83]{fitting-proof}
M.~Fitting.
\newblock {\em Proof Methods for Modal and Intuitionistic Logics}, volume 169
  of {\em Synthese Library}.
\newblock D. Reidel, Dordrecht, Holland, 1983.

\bibitem[Fit17]{fitting-dual-tableau}
Melvin Fitting.
\newblock Tableaus and dual tableaus.
\newblock In {\em Ewa Orlowska Volume}. 2017.

\bibitem[Gor08]{DBLP:conf/tphol/Gordon08} Mike Gordon.  
\newblock Twenty years of theorem proving for {HOL}s past, present and future.
\newblock In TPHOLs 2008 {\em Proceedings of Theorem Proving in
    Higher Order Logics, 21st International Conference}, LNCS~5170:1--5, 2008.

\bibitem[Kri59]{kripke-completeness}
Saul Kripke.
\newblock A completeness theorem in modal logic.
\newblock {\em Journal of Symbolic Logic}, 24(1):1--14, March 1959.

\bibitem[OGP11]{orlowska-joanna-book}
Ewa Orlowska and Joanna Goli{\'n}ska-Pilarek.
\newblock {\em Dual Tableaux: Foundations, Methodology, Case Studies},
  volume~33 of {\em Trends in Logci}.
\newblock Springer, 2011.

\bibitem[Sco93]{scott-computable}
Dana~S Scott.
\newblock A type-theoretical alternative to iswim, cuch, owhy.
\newblock {\em Theoretical Computer Science}, 1993.

\end{thebibliography}

\end{document}
